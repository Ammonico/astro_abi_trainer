\chapter{Merkmale guten Unterrichts}\label{GuterUnterricht}

\begin{uea}
	Beantworten Sie zunächst Q12 bis Q14 in \cref{Denk}.
\end{uea}

Woher weiß man denn, ob Unterricht gut ist? Woher weiß man, ob man als Lehrender bzw. Lehrende keinen Schaden 
anrichtet? Es ist schwer,  eine allgemeing{\"u}ltige Beschreibung f{\"u}r guten Unterricht zu erstellen. Eine einfache Arbeitsdefinition k{\"o}nnte darin
bestehen, dass ein Unterricht dann gut ist, wenn eine gr{\"o}{\ss}tm{\"o}gliche Anzahl an Sch{\"u}lern vorher
festgelegte Ziele erreicht. Somit h{\"a}ngt nun die Beurteilung des Unterrichts auch von der Definition der Ziele ab und von den Methoden, wie diese Ziele erreicht werden (siehe \cref{Lernziel} und \ref{Methoden})!

\begin{beisp2}
\begin{itemize}
\item
Ein sehr strenger Unterricht mit k{\"o}rperlicher Ma{\ss}regelung, wie zu Beginn des 20. Jahrhunderts durchaus {\"u}blich,
f{\"o}rdert das Auswendiglernen, behindert aber die freie Pers{\"o}nlichkeitsentwicklung.
\item
Freier Unterricht f{\"o}rdert die individuelle Pers{\"o}nlichkeitsentwicklung der Sch{\"u}ler, erbringt aber 
m{\"o}glicherweise weniger Durchschnitts- oder Spitzenleistungen als andere Unterrichtsformen.
\end{itemize}
\end{beisp2}

Sp{\"a}ter werden wir das wichtige Wechselspiel von Unterrichtszielen, didaktischen Prinzipien, Unterrichtskonzepten und 
methodischer Umsetzung kennen lernen. F{\"u}r eine zusammenfassende Darstellung der Merkmale guten Unterrichts sei
hier auf den PIKO-Brief Nr. 4 verwiesen \autocite{piko4}.

\bip\bip
\section{Die internationale Schulleistungsstudie PISA}
Das {\glqq}Programme for International Student Assessment{\grqq} (PISA) erfasst weltweit Sch{\"u}lerleistungen und vergleicht diese international. 
Initiator des Programms ist die OECD (Organisation f{\"u}r wirtschaftliche Zusammenarbeit und Entwicklung). Es werden 
 die Kompetenzen von 15-j{\"a}hrigen Jugendlichen beim Lesen, in der Mathematik
und den Naturwissenschaften europaweit erfasst. Sie wird alle drei Jahre erstellt. Die PISA-Studien werden als ein internationales 
Instrument angesehen, um alltags- und berufsrelevante Kenntnisse und F{\"a}higkeiten F{\"u}nfzehnj{\"a}hriger zu messen und vergleichbar zu machen.
PISA soll nicht nur eine Beschreibung des Ist-Zustandes liefern, sondern Verbesserungen ausl{\"o}sen. Zumindest  implizit wird der Anspruch 
erhoben, auf die nationalen Lehrpl{\"a}ne und Bildungsstandards zur{\"u}ckzuwirken.
\mip
Die Testaufgaben orientieren sich nicht an spezifischen Lehrpl\"{a}nen, sondern an Kompetenzen, die f{\"u}r den
Lernprozess und den Wissenserwerb wichtig sind. Naturwissenschaftliche Kompetenz beinhaltet, grundlegende 
naturwissenschaftliche Konzepte zu verstehen und mit naturwissenschaftlichen Denk- und Arbeitsweisen
vertraut zu sein ($\to$~\cref{KmK}).  

\bip\bip
\section{Basisdimensionen guten Unterrichts}

Es wurde,  der Versuch unternommen, u.a. und prominenterweise von Prof. Eckhard Klieme,\footnote{Eckhard Klieme (*1954), deutscher Bildungsforscher und Professor f{\"u}r Erziehungswissenschaft an der Johann Wolfgang Goethe-Universit{\"a}t Frankfurt am Main} aus der empirischen Unterrichtsforschung generische Grunddimensionen von Unterrichtsqualit{\"a}t zu entwickeln \autocite{Klieme}. Es lie{\ss}en sich drei Dimensionen voneinander abgrenzen:

\begin{itemize}
\item
Die erste Dimension \emph{Klassenf{\"u}hrung} fasst Skalen zur Regelklarheit, zu nachvollziehbaren Handlungsroutinen, zum Monitoring durch die Lehrkraft und zur St{\"o}rungspr{\"a}vention zusammen.

\item
Die zweite Dimension \emph{Sch{\"u}lerorientierung} f{\"u}hrt Skalen wie die Sensitivit{\"a}t f{\"u}r individuelle Bed{\"u}rfnisse, Unterst{\"u}tzung durch die Lehrkraft und die Abwesenheit von Leistungsdruck zusammen. In anderen Publikationen wird das als \emph{sch{\"u}lerorientiertes Unterrichtsklima} bzw. \emph{konstruktive Unterst{\"u}tzung} bezeichnet.

\item
Die dritte Dimension fasst unter dem Begriff \emph{kognitive Aktivierung} die Verwendung von Erkl{\"a}rungen und Aufgabenstellungen zusammen, die die SuS herausfordern, an vorhandenes Wissen anzukn{\"u}pfen und sich durch eigenst{\"a}ndiges Nachdenken neues Wissen zu erschlie{\ss}en.
\end{itemize}

Diese Basisdimensionen sind mittlerweile gut etabliert. Sie stimmen gut mit den international anerkannten Dimensionen {\glqq}Organisational Support{\grqq}, {\glqq}Emotional Support{\grqq} und  {\glqq}Instructional Support{\grqq} {\"u}berein.

\bip\bip
\section{Die IPN Interessenstudie Physik ({\glqq}Kieler{\grqq} Interessenstudie)}
Am Leibniz-Institut f{\"u}r die P{\"a}dagogik der Naturwissenschaften und Mathematik (IPN) in Kiel wurde eine Videostudie zur Beschreibung und Erkl{\"a}rung von Lehr-Lern-Prozessen im Physikunterricht erstellt. Die Ergebnisse des sechsj{\"a}hrigen Forschungsprojekts zeigen zum einen, wie einheitlich Physikunterricht in Deutschland hinsichtlich der Klassenorganisation, der Zielorientierung, der Lernbegleitung, der Fehlerkultur und der Experimente abl{\"a}uft. Zum anderen lassen die Analysen differentielle Effekte des Unterrichts auf Lernentwicklungen bei Sch{\"u}lerinnen und Sch{\"u}lern erkennen.
\mip
Wesentliche Ergebnisse lassen sich wie folgt zusammenfassen:
\begin{itemize}
\item
Kein Unterricht ist gleichwertig zu betrachten, jeder Unterricht hat St{\"a}rken und Schw{\"a}chen. Jeder Unterricht ist indviduell, jede Lehrkraft hat eigene {\glqq}Handschrift{\grqq}.
\item
 Experimente haben grosse Bedeutung, ca 70\% des Unterrichts werden vom Experiment bestimmt. Das Sch{\"u}lerexperiment ist zeitaufw{\"a}ndiger als das Demonstrationsexperiment. Es gibt signifikante Unterschiede zwischen den Lehrkr{\"a}ften. Sch{\"u}ler sind im Durchschnitt nur wenig bei Planung, Durchf{\"u}hrung, Auswertung von Experimenten beteiligt.
 \item
Unterricht ist typischerweise lehrerzentriert. Nur 17\% der Unterrichtszeit entfallen auf Sch{\"u}lerarbeitsphasen. Dominant ist das eng gef{\"u}hrte Klassengespr{\"a}ch im Stile des fragend-entwickelnden Verfahrens.
\item
Der Unterricht bietet nur wenig Gelegenheit f{\"u}r die aktive und eigenst{\"a}ndige Auseinandersetzung mit dem Stoff.
\item
Unterricht {\"u}ber {\glqq}klassische{\grqq} Inhalte dominiert; naturwissenschaftliche Denk- und Arbeitsweisen werden nur selten angesprochen.
\item
Viele Lehrkr{\"a}fte haben keine explizite Vorstellung,  wie Lernen funktioniert und welche Rolle sie beim Lernen einnehmen sollten. Z.~B. Rolle von Sch{\"u}lervorstellungen beim Lernen von Physik und zur F{\"o}rderung des Interesses kaum bekannt.
\end{itemize}


\bip\bip
\section{Die Hattie-Studie}
Der neuseel{\"a}ndische Bildungsforscher John Hattie erstellte eine international beachtete Studie zum Themenfeld {\glqq}Schulunterricht{\grqq}, 
welche er im Jahre 2009 in seinem Buch \emph{Visible Learning} vorstellte. In der Studie wurden die Ergebnisse von hunderten Metaanalysen zusammengef{\"u}rt,
welche aus vielz{\"a}ligen Lernstandserhebungen gewonnen wurden. Hattie stellte Einflussfaktoren auf Sch{\"u}lerleistungen und erfolgreiches Lernen zusammen. 
\mip
Zur Beurteilung von Einflussfaktoren auf den schulischen Lernerfolg ermittelte Hattie eine Effektst{\"a}rke, mit deren Hilfe er den Effekt einer 
bestimmten Ma{\ss}nahme auf den Lernerfolg bewertet. Hattie konnte somit darlegen, dass es weiterhin stark auf die Lehrperson ankomme, 
ob Sch{\"u}ler in der Schule erfolgreich sind.
\mip

Einige Beispiele:
\begin{itemize}
\item
{\emph{Sch{\"a}dlich f{\"u}r den Lernerfolg:}} Umzug, Krankheit, Fernsehen, Alleinerziehende Eltern, Sitzenbleiben, Sommerferien.
\item
{\emph{Was hilft nicht und schadet nicht:}} Offener Unterricht, Leistungsgruppierung, Interne Differenzierung, Web-basiertes Lernen, team-teaching.
\item
{\emph{Was hilft ein wenig:}} Reduzierung der Klassengr{\"o}{\ss}e, individualisiertes Lernen, Integration und Inklusion, Hausaufgaben, entdeckendes Lernen, induktives Unterrichten, regelm{\"a}{\ss}ige Leistungskontrollen, Zusatzangebote f{\"u}r Leistungsstarke.
\item
{\emph{Was hilft schon mehr:}} Angstreduktion, kooperatives Lernen, Kleingruppenlernen, peer tutoring, direkte Instruktion.
\item
{\emph{Was hilft richtig viel:}} Regelm{\"a}{\ss}ige Tests mit Feedback, metakognitive Strategien, Lehrkraft-Sch{\"u}ler-Verh{\"a}ltnis, Klarheit der Instruktion, Micro-Teaching, Akzelerationsprogramme, Formatives Assessment. 
\end{itemize}



