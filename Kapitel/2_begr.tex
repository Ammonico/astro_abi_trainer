\chapter{Begr\"{u}ndung von Physik in der Schule}\label{Begruendung}

Wie kann Physikunterricht gerechtfertigt ( = legitimiert) werden? \\
Ist es sinnvoll, Physik in der Schule zu unterrichten?

Unter welchen Gesichtspunkten ist diese Frage zu beantworten?
\begin{itemize}
	\item Aus der Sicht des Kindes?
	\item Aus der Sicht der Erziehenden?
	\item Aus der Sicht der Gesellschaft?
	\item Aus der Sicht der Wirtschaft?
\end{itemize}

\begin{enumerate}
	\item Kulturelle Identit\"{a}t

	\begin{enumerate}

		\item Lange Tradition einer Kultur in Europa, in Deutschland.

		\item Spezifisch naturwissenschaftliche Sichtweise:
		\begin{itemize}
			\item Naturwissenschaftliche Methode (Falsifikation von Hypothesen).
			\item Empirik (Experiment),
			\item Mathematisierung,
			\item Rationales Argumentieren,
			\item Exaktheit,
		\end{itemize}

		\item Entmythologisierung:
		\begin{itemize}
			\item \say{Die heilende Strahlkraft der Steine}
			\item Astronomie und Astrologie,
			\item Die teuflischen Handy-Strahlen.
		\end{itemize}

		\item Verantwortung f\"{u}r die Welt:
		\begin{itemize}
			\item Gestaltung der technischen Zivilisation
			\item Umwelterziehung: Kann ich anstelle einer Haushalts(Trocken-)Batterie auch ein Netzger\"{a}t verwenden?

		\end{itemize}


		\item Attribuierungen von Physik:
		\begin{itemize}
			\item Physik ist nicht nur die Technik-Hybris: Atombomben, Kraftwerke, Anonyme Apparate-Medizin,
			\item Ehrfurcht vor den Theorien der theoretisch-abstrakten Physik.
		\end{itemize}

	\end{enumerate}

	\item Lebensbew\"{a}ltigung

	\begin{enumerate}

		\item Handwerklich-technische Fertigkeiten, Berufsbildung
		\item Genaues Beobachten.

		\begin{itemize}
			\item In welcher Reihenfolge treten (welche) Farben im Regenbogen auf? In welcher Richtung ist der Bogen zu sehen?
		\end{itemize}


		\item Sprachliche Beschreibung:
		\begin{itemize}
			\item Stimmige Ausdrucksweisen: Der Strom flie{\ss}t, es liegt ein Spannung an,
			\item Bereicherung des Wortschatzes: El.\ Spannung, Druck, Temperatur, Verdampfen, Verdunsten, \dots
			\item Vertrautheit mit Einheiten.
		\end{itemize}

		\item Sicherheitsbewusstsein:
		\begin{itemize}
			\item Der F\"{o}hn in der Badewanne,
			\item Der Fotoapparat im Schwimmbad,
			\item Der Stuhl an der Wand,
		\end{itemize}

	\end{enumerate}

	\item Im Hinblick auf die Schule: Physik als \say{Methode}

	\begin{itemize}
		\item Farbe im Unterricht
		\item Spielerische Elemente,
		\item Handlungsorientierung,
		\item Soziale Lernziele: Gruppenexperiment,
		\item M\"{o}glichkeit zum Fach\"{u}bergriff:
		\begin{itemize}
			\item Mathematik: Gr\"{o}{\ss}enrechnen,
			\item Verkehrserziehung: Geschwindigkeit, Kr\"{a}fte, Fliehkr\"{a}fte, Bremswege.
		\end{itemize}

	\end{itemize}

	\item Weitere Gesichtspunkte:
	\begin{itemize}
		\item \"{A}sthetik,
		\item M\"{a}dchen und Physik,
		\item Entwicklung.
	\end{itemize}

\end{enumerate}