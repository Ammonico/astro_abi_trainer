\chapter{Denkanstöße}\label{Denk}

\begin{enumerate}[label=Q\arabic*:]
	\item
	Warum sollen Menschen (Sch\"{u}ler) Physik erlernen?
	\item
	Kann man \say{Physik unterrichten} lernen?
	\item
	K\"{o}nnen Jungen Physik besser verstehen bzw.\ lernen?
	\item
	Warum sollen im Physikunterricht Experimente durchgef\"{u}hrt werden?
	\item
	Warum ist Physik das --- mit Abstand --- unbeliebteste Schulfach?
	\item
	Ist Physikdidaktik eine Wissenschaft?
	\item
	Warum geht von gro{\ss}en Denkleistungen gerade der Physik eine fast unvergleichliche Faszination aus?
	\item
	Kann man Physik nur mit Hilfe von Mathematik verstehen?
	\item
	Ist die Wissenschaft Physik Fluch oder Segen f\"{u}r die Menschheit?
	\item
	Ist ein Lehrplan f\"{u}r das Unterrichten notwendig?
	\item Sind angesichts von Computer, Beamern und ChatGPT  noch andere Medien sinnvoll?
	\item Was sind für mich persönlich die zwei wichtigsten Merkmale guten Unterrichts?
	\item Wodurch wird meiner Meinung nach der Unterrichtserfolg am meisten gefährdet?
	\item Welche zwei Merkmale eines langfristig erfolgreichen Unterrichts könnten die empirischen Unterrichtsforscher als \say{Spitzenreiter} (Merkmale größter Einflussstärke) ermittelt haben?
\end{enumerate}

\begin{uea}
	Beantworten Sie diese Fragen für sich und diskutieren Sie Ihre Gedanken mit Ihrem Nachbarn!
\end{uea}