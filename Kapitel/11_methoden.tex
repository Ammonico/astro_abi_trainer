\chapter{Methodische Gestaltung des Unterrichts}\label{Methoden}

In diesem Kapitel werden zentrale Aspekte der methodischen Gestaltung von Unterrichtseinheiten besprochen. Dazu geh\"{o}ren Handlungs- oder Aktionsformen, Sozialformen und Organisationsformen. Diese bestimmen, wie Lernprozesse konkret ablaufen und wie die Interaktion zwischen Lehrenden und Lernenden sowie unter den Lernenden selbst organisiert wird. Diese Formen bieten Lehrkr\"{a}ften Werkzeuge, um den Unterricht abwechslungsreich, effektiv und sch\"{u}lerzentriert zu gestalten.

\begin{itemize}
\item
\emph{Handlungs- oder Aktionsformen}  beziehen sich auf die Art und Weise, wie die Lernenden aktiv im Unterricht t\"{a}tig werden. Beispiele sind Experimentieren, Diskutieren, Pr\"{a}sentieren oder Probleml\"{o}sen. Diese Formen geben vor, welche Art von Aktivit\"{a}t im Fokus steht und welche Handlungskompetenzen der Sch\"{u}ler gef\"{o}rdert werden.
\item
\emph{Sozialformen} definieren die Gruppierung der Lernenden im Unterricht und bestimmen, wie die Lernenden zusammenarbeiten. Die wichtigsten Sozialformen sind Einzelarbeit, Partnerarbeit, Gruppenarbeit und Klassenunterricht. Sie legen fest, wie die soziale Interaktion und Zusammenarbeit im Lernprozess organisiert wird.
\item
\emph{Organisationsformen} betreffen die organisatorische Struktur und den Rahmen des Unterrichts, einschlie{\ss}lich Raum- und Zeitgestaltung, Materialeinsatz und Klassenmanagement. Dazu geh\"{o}rt auch die Planung von Unterrichtseinheiten, Projekten oder Exkursionen. Organisationsformen legen fest, wie der Unterricht praktisch umgesetzt wird.
\end{itemize}

%-----------------------
\bip\bip
\section{Handlungs- oder Aktionsformen}

\subsection{Vortrag}

Lerninhalte werden durch Vortrag --- monologisch ---
dargeboten oder mitgeteilt.

\begin{beisp}
	\begin{itemize}
\item Lehrervortrag,
\item Sch\"{u}lerreferat,
\item Reinform: Vorlesung,
\item Kongress-Vortr\"{a}ge.
\end{itemize}
\end{beisp}


Diese Methode zeichnet sich dadurch aus, dass sie
einen schnellen, dichten, \"{o}konomischen Unterricht erm\"{o}glicht.
Er unterliegt stark der Steuerung und Kontrolle des Vortragenden,
Fehlinformationen k\"{o}nnen von vornherein vermieden werden.

\bip
Die Vorbereitung ist festgef\"{u}gt, sie erfolgt gem\"{a}{\ss} eines
(realen oder virtuellen) Skripts.

\bip
Heute: Unter Umst\"{a}nden stark begleitet und gepr\"{a}gt durch
Einsatz von AV-Medien. Sie werden zunehmend abgel\"{o}st durch
Pr\"{a}sentationen mit Laptop und Beamer (,,Powerpoint'').


\subsection{Lehrgespr\"{a}ch}
impulsgebend --- gef\"{u}hrt/gelenkt --- Frage- und Antwortspiel sokratisch


%-----------------------
\bip\bip
\section{Sozialformen}\label{Sozial}

Sozialformen beschreiben das Auftreten der Klasse in sozialen
Gruppen.
Sie sind zun\"{a}chst als unabh\"{a}ngig von Unterrichts- oder
Organsiations\-formen zu sehen.

Man unterscheidet die ,,Arbeit'' \dots
\begin{itemize}
\item in Gro{\ss}gruppen (B: Schulfest)
\item im Klassen\-verband, lehrerzentriert (B: Frontalunterricht).
\item im Klassen\-verband, gruppendynamisch (B: Sitzkreis, Projekt).
\item in Kleingruppen (3 -- 6 Kinder). (B: Sch\"{u}lerexperimente)
\item in Partnerschaft (2 Kinder). (B: Hausaufgabenvergleich)
\item Einzeln (allein, individualisiert)
(B: Stillarbeit, Hausaufgabe, Einzelunterricht)
\end{itemize}

In der Physikdidaktik sind  diese Sozialformen u.~a. interessant
im Hinblick auf das Experimentieren, vgl.\ dort.


\subsection{Arbeit im Klassenverband}

Arbeit im Klassenverband ist
\begin{itemize}
\item ist in der Praxis vorherrschend,
\item erm\"{o}glicht eine umfassende und \"{o}konomische
Wissensvermittlung ($\to$ Darbietender Unterricht),
\item erfordert einen nicht so gro{\ss}en Planungsaufwand,
\item kann leichter gesteuert werden,
\item ist bei der Durchf\"{u}hrung relativ anstrengend,
\end{itemize}

% Zitat: Comenius (in \cite[Kap. 19, S.\ 126/127]{Reble2}). ?????


\subsection{Arbeit in Kleingruppen}

\begin{itemize}
\item Die Erreichung sozialer Lernziele
(F\"{a}higkeit zu Teamarbeit, Kooperation,
Kommunikation, Konfliktl\"{o}sung) wird unterst\"{u}tzt.
(Unter Umst\"{a}nden tritt genau das Gegenteil ein.)
\item Viele Unterrichtsprinzipien (Handlungsorientierung,
Anschauung, Unmittelbarkeit) lassen sich leichter umsetzen.
\end{itemize}

Bez\"{u}glich der Arbeitsauftr\"{a}ge unterscheide:
\begin{itemize}
\item
Parallel identisch (arbeitsgleich) (B Aufbau einer Schaltung), ein
Wettbewerbscharakter kann entstehen.
\item
Parallel abwechselnd (B Arbeitsplatz-Praktikum).
\item
Parallel erg\"{a}nzend (arbeitsteilig) (B Experimentalpraktikum hier).
\item
Frei
\end{itemize}

Es besteht grunds\"{a}tzlich das Spannungsfeld zwischen
\begin{quote}
Eng gef\"{u}hrter geplanter disziplinierter \q und \\
Freier von Eigeninitiative getragener Gruppenarbeit
\end{quote}


Globale Vorbereitungen (F\"{u}r das ganze Schuljahr)
\begin{itemize}
\item
Gruppenbildung: Es besteht die M\"{o}glichkeit, ein bestehendes
Sozialgef\"{u}ge etwas aufzuweichen.
\begin{itemize}
\item
In leistungsm\"{a}{\ss}ig homogenen Gruppen
bleibt nicht alles den K\"{o}nnern \"{u}berlassen, schwache
Sch\"{u}ler sind gefordert und k\"{o}nnen sich nicht einfach auf
das Zuschauen (Rezeption) beschr\"{a}nken.
\item
In leistungsm\"{a}{\ss}ig heterogenen Gruppen k\"{o}nnen versierte
Sch\"{u}ler schwierigere Situationen besser meistern und so ihre
Gruppenpartner betreuen oder zumindest mitziehen.
\item
Sch\"{u}ler entscheiden frei.
\item
Einteilung nach Interesse.
\item
Einteilung nach sozialen Gesichtspunkten (vgl.\ soziale Lernziele).
\end{itemize}
\end{itemize}

%Besondere Beispiele: Lernzirkel,


\subsection{Partnerarbeit}

\begin{itemize}
\item Durchf\"{u}hren von Experimenten
\item L\"{o}sen von Rechenaufgaben
\item Gegenseitiges Erkl\"{a}ren und Lehren von Konzepten
\item Erstellen eines Modells
\item Durchf\"{u}hrung, Diskussion und Analyse von Experimenten
\item Erstellen von Plakaten oder Pr\"{a}sentationen
\item Dialogisches Lernen
\item Gemeinsames Recherchieren
\item Peer-Feedback bei Experiment-Berichten
\end{itemize}


\subsection{Einzelarbeit}

\begin{itemize}
\item Extremform bei der Binnendifferenzierung.
\item Evaluation: Lehrer kann sich ein Bild vom individuellen
Leistungsstand  machen.
\item Diskussion und Anregung durch Partner fehlen.
\item Individuelle Neigungen und Leistungssverm\"{o}gen k\"{o}nnen viel
leichter ber\"{u}cksichtigt werden.
\end{itemize}

\begin{beisp}
	\begin{itemize}
		\item Programmierter Unterricht,
		\item Nachhilfe-Einzelunterricht,
		\item Stillarbeit, 
		\item Freiarbeit (nach Anweisungen),
		\item Hausaufgabe.
	\end{itemize}
\end{beisp}

%-----------------------
\bip\bip
\section{Organisationsformen}\label{Orga}

In \cite[S.\ 97]{DuitHausslerKircher} tritt der Begriff
,,Methodenkonzeptionen'' auf.
In \cite{BleichrothDahnckeJung} sind mit Organisationsformen
die Sozialformen gemeint.

\mip Hochsprungbild: Ein Hochsprung wird in verschiedenen
Rahmen (Spielplatz, Training, Sportfest, Wettkampf) durchgef\"{u}hrt.


\subsection{Klassen- und Fachlehrerprinzip}
\begin{itemize}
\item Klassenlehrerprinzip: Eine Klasse hat einen Lehrer, der
sie im wesentlichen in allen F\"{a}chern unterrichtet.
Vorteile:
\begin{itemize}
\item Es werden soziale Lernziele betont.
Es kann sich leichter auch eine emotionale pers\"{o}nliche
Beziehung bilden.
\item Die Schulorganisation wird einfacher: Stundenplan, inhaltliche
Abstimmung, Leistungserhebung, Hausaufgaben, Termine, Zeugnisse.
\item Der Unterricht kann zeitlich flexibler gestaltet werden.
\item Es k\"{o}nnen leichter alternative Organisationsformen des
Unterrichts (Projekte, f\"{a}\-cher\-\"{u}ber\-grei\-fen\-der Unterricht,
Freiarbeit, au{\ss}erschulische Lernorte\dots) durchgef\"{u}hrt werden.
\item Die Verantwortung hinsichtlich Leistungsbewertung,
Schullaufbahn liegt bei einer Person.
\item Die Begleitung und Beratung des Sch\"{u}lers und der
Eltern geschieht durch eine Person.
\end{itemize}

\item Fachlehrerprinzip:
Eine Klasse wird in verschiedenen F\"{a}chern von
verschiedenen Lehrern unterrichtet.
Organisatorische Aufgaben \"{u}bernimmt der Klassleiter.
Vorteile:
\begin{itemize}
\item Es wird die fachliche Kompetenz betont.
\item Hinsichtlich Leitsungsbewertung und Schullaufbahn liegt die
Verantwortung bei einer Gruppe von Lehrern (Team).
\item Die Auswirkungen einer
,,schwierigen Lehrerpers\"{o}nlichkeit'' sind nicht so drastisch.
\end{itemize}
\item Mischformen: Zwei Lehrer
(Sprachlich-gesellschaftswissenschaftlicher
und natur\-wis\-sen\-schaft\-lich-technischer Bereich)
unterrichten zwei Klassen.
Dies ist beispielsweise in Waldorfschulen (vgl.\ Prospekt)
verwirklicht.
\end{itemize}


\subsection{Zeitstruktur des Unterrichts}

\begin{itemize}
\item Der Einzelfachunterricht erfolgt im klassischen
45-Minuten-Takt, dies stellt eine starke Einengung
im Hinblick auf freiere Unterrichtsgestaltung dar.
\item Im Blockunterricht wird ein einzelnes Fach \"{u}ber einen
l\"{a}ngeren Zeitraum massiv unterrichtet.
\item Im Epochalunterricht (teilweise beispielsweise in
Waldorfschulen) werden einzelne F\"{a}cher \"{u}ber einen
l\"{a}ngeren Zeitraum unterrichtet und dann eine Zeit lang gar nicht.
\begin{itemize}
\item Dies setzt nicht notwendig ein Klasslehrerprinzip voraus.
\item So wird eine Konzentration auf das Fach, ein tieferes
Eindringen in
eine bestimmte Thematik erm\"{o}glicht.
\item Es kann organisatorisch variiert werden (Projekte,
au{\ss}erschulische Lernorte).
\item Das unnat\"{u}rliche Stundenkorsett weicht einer
flexiblen Unterrichtsgestaltung.
\item Unter Umst\"{a}nden tritt eine gewisse Eint\"{o}nigkeit auf.
\end{itemize}
\item
Freiarbeit:
\begin{itemize}
\item Sie gesteht den Sch\"{u}lern Eigeninitiative und
Eigenverantwortung zu,
\item Es soll ein festes ,,Auftrags-Pensum'' erf\"{u}llt werden.
\item Sie wird verwirklicht durch Wahlangebote
(Materialtische, Lerntheken), die
au{\ss}erhalb oder neben dem regul\"{a}ren Unterricht
wahrgenommen werden.
\item Der Lehrer tritt als Anbieter, Organisator und Berater auf.
\end{itemize}

\item Kurs:
\begin{itemize}
\item Ein abgegrenztes Sachgebiet wird systematisch behandelt.
Die Auswahl der Lerninhalte ist im
allgemeinen an einer Fachsystematik orientiert.
(Steil- oder Crashkurs).
\item Es werden eher kognitve als soziale Lernziele betont.
\item Vorwiegende Aktionsform ist der Frontalunterricht.
\item In Kurssystemen werden Wahlentscheidungnen m\"{o}glich
(Vgl.\ Kollegstufe der Gymnasien).
\item Beispiele: Erste-Hilfe-Kurs, Fahrschule, Wickelkurs,\dots
\end{itemize}
\end{itemize}


\subsection{Au{\ss}erschulische Lernorte}

Exkursionen, beispielsweise im Rahmen von Wandertagen,
Projekttagen oder -wochen.

\begin{beisp}
	\begin{itemize}
\item Museen (z.~B. Zukunftsmuseum Nürnberg), Ausstellungen.
\item Technische Betriebe
(Kraftwerke, Betriebe der Stromversorgung).
\item Forschungseinrichtungen (Garching).
\item Planetarium, Sternwarte.
\end{itemize}
\end{beisp}



\subsection{Projekt --- projektorientierter Unterricht}

Wortbedeutung lat: Der Begriff geht auf das lateinische Verb
,,proicere'' zur\"{u}ck.
Die deutsche Bedeutung ist wesentlich ,,vorwerfen'' oder
,,hinausragen'', erst im \"{u}bertragenen Sinne ,,entwerfen''.

Historischer Abriss
\begin{itemize}
\item John Dewey (1859 -- 1952),
schlagwortartiger Umriss seines Denkens und seiner Wirkung:
\begin{itemize}
\item John Dewey ist ein Vertreter des amerikanischen Pragmatismus:
Der Wert einer Wahrheit, einer Erkenntnis der Welt, ist im Ziel, das mit ihr
erreicht werden soll, begr\"{u}ndet.

\item Erfahrungsorientiertheit:
\begin{quote}
Ein Gramm Erfahrung ist besser als eine Tonne Theorie, einfach deswegen, weil jede
Theorie nur in der Erfahrung lebendige und der Nachpr\"{u}fung zug\"{a}ngliche Bedeutung hat.
\end{quote}

\item Handlungstheoretischer Ansatz: ,,Learning by doing''.

\item Er ist Begr\"{u}nder einer an der Universit\"{a}t von Chicago angegliederten
Laborschule: Umsetzung der p\"{a}dagogischen Theorien in die Praxis.

\item Er bahnte dem Arbeitsunterricht in USA den Weg (In D: Kerschensteiner).

\item William H.\ Kilpatrick ist der herausragendste Vertreter in der Folgezeit,
er konkretisiert Deweys Ideen weiter und r\"{u}ckt dabei die Lern- und Schulpraxis in
der Vordergrund.
Von ihm stammt auch das Projekt-Ablaufschema (siehe unten), das zu einer
leichteren Handhabbarkeit, aber auch zu einer Art Formal-Verselbstst\"{a}ndigung
der Projektidee beitrug.

\end{itemize}

\item Rezeption in Deutschland:
\begin{itemize}
\item Reformbewegung der 20er Jahre.
\item Amerikanischer Einfluss in den 50er Jahren.
Projektorientierung ist gekoppelt an demokratische Entwicklung und Erziehung.
\item In den 60er/70er Jahren wird die Projektidee zum Bestandteil der
politischen aufbrechenden Studenten\-bewegung.
Die gesellschaftlich-politische Relevanz wird zum
Charakteristikum der Projektidee.
\end{itemize}

\item Heute: Wieder starker Bestandteil der p\"{a}dagogischen Diskussion im Zuge
einer ,,Alltagswende'', einer tendenziellen Abwendung von einer
vermeintlich zu stark wissenschafts- und theoriebezogenen Auffassung \"{u}ber Unterricht
und Erziehung. Ein Hauptvertreter ist Herbert Gudjons, er kritisiert die
faktische Herabw\"{u}rdigung der Projektidee zum ,,Schul-Event''.
\end{itemize}

\pph{Merkmale eines Projekts}
Es liegt insgesamt der Gedanke des ,,Umfassenden'' und der
,,Integration'' zugrunde.
\begin{itemize}
\item
Es geht um eine umfassende und echte Aufgabe.
\item
Das Projekt findet \"{u}ber einen l\"{a}ngeren Zeitraum statt.
(F\"{u}r ein klasseninternes Projekt ist also der
Epochalunterricht g\"{u}nstig.)
Gegenw\"{a}rtig werden Projekte innerhalb von Projektwochen
oder -tagen verwirklicht.
\item
Das Projekt findet in einem fach\"{u}bergreifenden
(interdisziplin\"{a}ren) Rahmen statt.
\item
Es ist bezogen auf die Bed\"{u}rfnisse, die aktuelle Situation, die
Lebenswelt, die    Interessen der Sch\"{u}ler.
\item
Sch\"{u}ler sind stark --- in allen Phasen --- beteiligt.
Im Idealfall wird ein Projekt von den Sch\"{u}lern initiiert, geplant
und ausgef\"{u}hrt.
Der Lehrer tritt begleitend-beratend auf.
\item
Sch\"{u}ler und Lehrer artikulieren ihr gemeinsames Interesse.
\begin{quote}
Dass Sch\"{u}ler und Lehrer an ungel\"{o}sten Problemen gemeinsam
arbeiten, ist offenbar die erzieherischste aller Bem\"{u}hungen der
Schule (W.H.\ Kilpatrick).
\end{quote}
\item
Soziale Lernziele sind stark betont (Gruppenprozesse).
\item
Gesellschaftlich-politische Relevanz.
\item
Produkt- (oder Handlungs-)orientierung (B: Erstellung einer Homepage).
\item
Ganzheitlicher Ansatz (Alle Sinnesorgane, alle Ebenen des
menschlichen Agierens: affektiv, enaktiv, kognitiv --- Herz,
Hand und Kopf).
\item
Herstellung von \"{O}ffentlichkeit.
\end{itemize}

\pph{Zeitliche Phasung eines Projekts}

Von W.\ Kilpatrick stammt die klassische Phasung in
\begin{quote}
Purposing --- Planning --- Executing --- Judging.
\end{quote}

Es gibt die unterschiedlichsten konkreteren Ausformungen, als Beispiel sei
aufgef\"{u}hrt:
\begin{enumerate}
\item
Inititative (mit Skizze).
\item
Planung: Schriftliche Fixierung, Arbeitsteilung, Verantwortung,
(Finanzierung).
\item
Organisation, Logistik: Raum, Zeit, Materialien, Absprachen.
\item
Durchf\"{u}hrung (mit Darbietung des Produkts).
\item
Reflexion, Nacharbeit.
\end{enumerate}

Pers\"{o}nliches Projekt, Kleingruppenprojekt, Klasseninternes Projekt,
Schulprojekt.

\pph{Grenzen und Probleme bei der Umsetzung}

\begin{itemize}
\item
Traditionelles Verst\"{a}ndnis von Unterricht und Erziehung bei Lehrern, Sch\"{u}lern,
Administration, politischer Gestaltung.
Humbold'sches Bildungsideal,
Bedeutung von abstrakter Theorie in den Wissenschaften.
\item
Fehlende institutionelle Voraussetzungen: Unterrichtliche Schemata.
\item
Fehlende logistische Voraussetzungen: Sachen, Finanzen, R\"{a}ume.
\item
Fehlende Zeit der Sch\"{u}ler, der Lehrer, in Stundenpl\"{a}nen.
\end{itemize}

\pph{Abmilderung: Der projektorientierte Unterricht}

G.\ Otto, 1974: Angesichts der schwierigen Rahmenbedingungen ist
der Idealtyp eines Projekts kaum zu verwirklichen.
Projekte werden mehr oder weniger fachspezifisch in den
fachorientierten Schulunterricht eingebunden.
Der Lehrer hat eine st\"{a}rker pr\"{a}gende Rolle inne.


\pph{M\"{o}gliche physikalisch orientierte
Projektthemen}

\begin{itemize}
\item
Fahrrad, Mofa, Motorrad, Auto (Der Pauker).
\item
(Papier-)Flugzeuge, Weltraumfahrt.
\item
Dampfboot, Dampfmaschine.
\item
Bus, Bahn, Strassenbahn, U-Bahn.
\item
Elektronik-Spielzeug: Elektronischer W\"{u}rfel, MorseApparat.
\item
Diskothek: Lichtorgel, Stroboskop, Lautsprecher.
\item
GameBoy, Computer.
\item
Physikalisches Spielzeug:
Reibstock, Reibspecht, keltischer Wackelstein.
\item
Darda-Autos.
\item
Optik: Zauberspiegel.
\item
Energie: Krafwerke, Energieversorgung, Energiesparen.
\item
Solartechnik: Bau eines Sonnenspiegels, Solarzellen,\dots
\item
Sport: Bumerang, Skifahren, Schwimmen, Skateboard, W\"{u}rfe.
\item
Astronomie:
Bau eines Fernrohrs, Beobachtung von Planeten, Sternen,
Finsternissen.
\item
Musik: Kl\"{a}nge, Musikinstrumente.
\item
Wetter:
\item
Umwelt: Wasserschutz, L\"{a}rmschutz (\"{O}l-Recycling).
\item
Schulleben: Cafe, Sauberes Schulhaus, Schulhof, Klassenzimmer.
\item
Schulzeitung,\dots
\end{itemize}


\begin{beisp}
	M\"{o}gliche inhaltliche Ideen f\"{u}r die Umsetzung des Projekts \textbf{\say{Fahrrad}}
	\begin{itemize} %%%%%%%%%%%
	\item
	Bewegung: Geschwindigkeit, Beschleunigung, Bremsweg.
	Tachometer (Kreiszahl $\pi$).
	\item
	Reibung, Bremsen (Felgen-, Trommelbremse),
	Erw\"{a}rmung beim Bremsen.
	\item
	El.\ Strom, Stromkreis, Leitf\"{a}higkeit (Kontakt, Rost), Batterien,
	Dynamo, Gl\"{u}hbirne, LED.
	\item
	Kraftwandlung: Hebelgesetz, \"{U}bersetzung bei: Gangschaltung(stypen),
	Bremsen (Felgen, Trommel).
	\item
	Drehimpuls: Warum f\"{a}llt ein Fahrrad beim Fahren nicht um?
	\item
	Pflege, Wartung: Werkzeuge (Gegenkraft), Kontermutter,
	Bedienung, \"{O}len-Schmieren, Reifen-Flicken, Rost.
	\item
	Beleuchtung: Birnchen, Fassung, Katzenauge
	(Tripelspiegel), R\"{u}ckspiegel.
	\item
	Fahrradbau: Werkstoffe (stabil -- leicht),
	Kette, Speichen, Kugellager,
	\item
	Federn an Mountainbike, Sattel, Kettenschaltung.
	\item
	Physikalische Gr\"{o}{\ss}en (Sachrechnen): Gewicht, L\"{a}ngen,
	Geschwindigkeit.
	\item
	Fahrradfarhen zur Schule, in der Stadt, \"{u}ber Land
	($\to$ Geographie):
	Route, Fahrtzeit, Entfernungen, Steigungen, Fahrradwege.
	\item
	Gesundheit ($\to$ Sport), Kraft, Ausdauer, Geschicklichkeit
	(Hochrad), Doping.
	\item
	Sicherheit ($\to$ Verkehrserziehung): Tragen eines Helmes,
	Freih\"{a}ndig fahren,
	gleichm\"{a}{\ss}ige Beladung, Bewegungsfreiheit, Fairness.
	\end{itemize}
	
	M\"{o}gliche methodische Ideen f\"{u}r die Umsetzung des Projekts
	\begin{itemize}
	\item
	Fahrradtag.
	\item
	Wettbewerbe (Geschicklichkeit, Wissen, Verkehrssicherheit).
	\item
	Reparaturdienst.
	\item
	Besichtigung einer Fahrradwerkstatt.
	\item
	Auswerten von Unfallstatistiken, Testberichten, Sportberichten
	(Tour de France).
	\item
	Fahrt (Wandertag, zum Schullandheim).
	\end{itemize}

\end{beisp}



\bip\bip
\section{Lernen an Stationen}

\emph{Lernen an Stationen} (syn: Stationenlernen, Lernparcour) beschreibt den organisatorischen Rahmen f\"{u}r
das Lernen einer Gruppe. Das wesentliche Element sind \emph{Stationen}: Inhaltlich und r\"{a}umlich abgetrennte Einheiten, die von den Lernenden besucht und dabei bearbeitet werden.
\mip
Die Idee geht auf das Zirkeltraining (circuit training) innerhalb des Sportunterrichts zur\"{u}ck.
In der Diskussion um Ver\"{a}nderung der Unterrichtskultur wird das Stationenlernen
in das Umfeld des ,,Offenen Unterrichts'' eingeordnet.
\mip
Deshalb sind auch allgemeinere Prinzipien und Zielsetzungen aus dem Offenen Unterricht
mit dem Stationenlernen verbunden:
\begin{itemize}
\item Selbstst\"{a}ndigkeit,
\item Eigenorganisation und -reflexion des Lernprozesses, ,,das Lernen lernen'',
\item Soziale Lernziele: Kommuniktaion, Teamf\"{a}higkeit, Kooperationsbereitschaft, Gruppendynamik,
\item Lernen mit allen Sinnen: Optische, akustische, taktile, motorische Auseinandersetzung mit den Inhalten,
\item Lernen in allen menschlichen Dimensionen: Lernen mit Kopf, Herz und Hand,
\item Affektive Ziele: Freude am Lernen, Spielerische Auseinandersetzung,
\item Handlungsorientierung: Handwerkliche Fertigkeiten, ,,learning by doing'',
\item Fach\"{u}bergreifende Ans\"{a}tze.
\end{itemize}

\mip
Wesentlich ist eine gute, genaue und umfassende Planung und Organisation des Stationenlernens hinsichtlich
organisatorischem Rahmen, inhaltlicher Gestaltung, methodischer Vielfalt.
\mip
Bei der Durchf\"{u}hrung ergibt sich f\"{u}r den Lehrer die M\"{o}glichkeit, \dots
\begin{itemize}
\item individuelle Probleme oder (kollektive) Fehlermuster zu erkennen, Lernerfolge festzustellen,
\item Lernerfolge festzustellen, evtl.\ zu kontrollieren,
\item individuelle Beratung oder F\"{o}rderung anzubieten.
\end{itemize}
\mip
Eine Aufarbeitung im Nachhinein er\"{o}ffnet eine Wiederholung und Verbesserung des Stationenlernens.
\mip
Die \"{a}u{\ss}ere Organisation des Stationenlernens weist vielf\"{a}ltige Variationsm\"{o}glichkeiten auf:
\begin{itemize}
\item Der Besuch der Stationen erfolgt
\begin{itemize}
\item einzeln,
\item paarweise,
\item in (kleineren) Gruppen.
\end{itemize}

\item Die r\"{a}umliche Anordnung ist unter Umst\"{a}nden nicht realisiert, sie legt also evtl.\ nur den
zeitlichen Ablauf nahe.
\begin{itemize}
\item \emph{Lernzirkel}: zyklische Anordnung der Stationen,
\item Lernstra{\ss}e: Lineare Anordnung der Stationen,
\item Doppelzirkel: Anordnung in zwei (konzentrischen) Kreisen, Fundamentum und Additum,
\item Lernspirale.
\end{itemize}

Die Stationen sollten Kennzeichen (durch Namen, Nummern oder Buchstaben) haben, die diese Anordnung widerspiegeln.

\item Man kann Stationen einrichten, die in Bezug auf verschiedene Kategorien er den \"{a}u{\ss}eren Ablauf verschiedene
Funktionen erf\"{u}llen:
\begin{itemize}
\item in Bezug auf den Lernprozess:
\begin{itemize}
\item Selbstst\"{a}ndiges Erarbeiten,
\item \"{U}bung,
\item Wiederholen,
\item Vertiefung.
\end{itemize}

\item \"{A}u{\ss}erer Ablauf:
\begin{itemize}
\item Kernstation: Sie geh\"{o}rt zum ,,Kern'' des Stationenlernens.
\item R\"{u}ckzugsstation: Beispielsweise der eigene Platz des Sch\"{u}lers.
\item Parallelstationen: Aus organisatorischen Gr\"{u}nden enthalten einige Stationen gleiche Inhalte oder Auftr\"{a}ge.
\item Pufferstation: Da die Verweildauer bei verschiedener Stationen i.a.\ unterschiedlich ist, ist
ein Ausweichen m\"{o}glich.
\item Pflicht- oder Wahlstation.
\end{itemize}

\item Medieneinsatz:
\begin{itemize}
\item Arbeitsbl\"{a}tter,
\item AV-Medien aller Art,
\item Computer-Einsatz,
\item Begegnung mit dem Lerngegenstand: Betrachtung, Experimente,
\item F\"{u}r \"{U}bung: Lernkarteien, Vielf\"{a}ltige Spielformen, evtl.\ mit (leichtem) Wettbewerbscharakter.
\end{itemize}
\end{itemize}

\item Die Stationenauswahl oder -abfolge \dots
\begin{itemize}
\item ist (durch Auftr\"{a}ge) geplant und vorgegeben, evtl.\ kontrolliert,
\item unterliegt \"{a}u{\ss}eren Bedingungen oder einer geeigneten Lernabfolge,
\item ist frei ausw\"{a}hlbar ($\to$ Freiarbeit),
\item kann Erfordernissen der Differenzierung (hinsichtlich Niveau oder Neigung) folgen.
\end{itemize}

Der Wechsel erfolgt frei oder nach Aufforderung --- beispielsweise durch einen Gongschlag.
\item
Konkrete Durchf\"{u}hrung in mehreren Phasen:
\begin{itemize}
\item Im Gespr\"{a}ch wird das Thema eingef\"{u}hrt.
\item Vorstellung der Stationen: Evtl.\ Rundgang oder Erl\"{a}uterung anhand eines Ablaufplans
(f\"{u}r die Sch\"{u}lerhand: Laufzettel).
\item Eigentliches Lernen an den Stationen.
\item Abschlussgespr\"{a}ch im Plenum.

\end{itemize}

\end{itemize}

\bip\bip
\section{Fach\"{u}bergreifender Unterricht}
\subsection{Begriffe --- Alternativen}

Fach- oder f\"{a}cher\"{u}bergreifender Unterricht wird auch synonym
oder \"{a}hnlich als fachintegrierender, fachverbindender oder
interdisziplin\"{a}rer Unterricht bezeichnet.

\mip
Speziell im naturwissenschaftlichen Bereich spricht man auch vom
integrierten naturwissenschaftlichen Unterricht.

Je nach Intensit\"{a}t des Anspruchs von Integration unterscheidet man
fach\"{u}bergreifenden Unterricht, der organisiert ist
\begin{itemize}
\setlength{\itemsep}{0mm}
\item
vom Einzelfach her:
Hier wird verst\"{a}rkt auf Querverbindungen zu Nachbar-,
Anwendungs- oder Spezialdisziplinen eingegangen.
Unter Umst\"{a}nden geschieht eine Abstimmung
bzgl.\ der parallel unterrichteten F\"{a}cher.
Organisatorisch geschieht dies beispielsweise durch
Querverweise im Lehrplan.
\item
im F\"{a}cherverbund: beispielsweise mit anderen Naturwissenschaften, Sport, \dots
\item
als eigenst\"{a}ndige Institution:
Beispielsweise als \say{grundlegender
(naturwissenschaftlicher bzw.\ )
gesellschaftswissenschaftlicher bzw.\ musischer Unterricht} oder
unter eigenem Leitthema (Kybernetik, ITG, Projektthema,\dots).
\end{itemize}

\subsection{Wissenschaftstheoretische Aspekte}

\begin{itemize}
\setlength{\itemsep}{0mm}
\item
Der der heutigen westlichen Kultur innewohnenden Polarit\"{a}t
zwischen einer
musisch-geistes\-wis\-sen\-schaft\-lich-humanistischen
Ausrichtung und einer technisch-naturwissenschaftlich-rationalen
Ausrichtung (C.P.\ Snow) wird entgegengearbeitet.
\item
Die stark vernetzten und komplexen Probleme der heutigen Zeit
k\"{o}nnen nur durch ganz\-heit\-lich-\"{u}ber\-grei\-fen\-de
Denkans\"{a}tze in geeigneter Weise angegangen werden.
\item
Eine Vergleichsschau der eigenen Fachdisziplin
mit anderen Fachdisziplinen erm\"{o}glicht ein Aufsp\"{u}ren von
Wechselwirkungen, Analogien, gemeinsamen L\"{o}sungen, neuen
Denkans\"{a}tzen,\dots
\item
Naturwissenschaftliche Inhalte werden als \say{genuin die
Natur beschreibend} empfunden, wenn sie nicht als durch
ein Teilgebiet allein definiert erfahren werden.
\end{itemize}

\subsection{Unterrichtsprinzipien}
Fach\"{u}bergreifender Unterricht versucht,
Inhalte von einem Sachthema oder Themenbereich her aufzugreifen,
er ist nicht so sehr an dem Fachkanon eines Faches orientiert.

Aus der Themenorientierung ergibt sich die Betonung einer
ganzen Reihe von aktuell eingeforderten Unterrichtsprinzipien:
\begin{itemize}
\setlength{\itemsep}{0mm}
\item
Lebensn\"{a}he, Alltagsn\"{a}he, Heimatn\"{a}he:
Er tr\"{a}gt st\"{a}rker den emotionalen, lebenspraktischen Bed\"{u}rfnissen
von Sch\"{u}lern Rechnung.
\item
Aktualit\"{a}t.
\item
Ganzheitlicher (auch: interdisziplin\"{a}rer) Ansatz :
Es wird aus dieser Sicht ein
,,Lernen am Gegenstand bzw.\ am Sachthema'' unterst\"{u}tzt.
Prinzipien wie Veranschaulichung, handelnder Unterricht,
Produktorientierung werden unterst\"{u}tzt.
\item
Das Prinzip einer Wissenschaftsorientierung und der Anspruch an
wissenschaftlicher Systematik, Vollst\"{a}ndigkeit (eines Fachkanons)
oder Analytik tritt in den Hintergrund.
\end{itemize}

Viele dieser Gesichtspunkte sind auch im projektorientierten
Unterricht aktuell.

\subsection{Lerntheoretische Aspekte}
\begin{itemize}
\setlength{\itemsep}{0mm}
\item
Ein Lernen im System (durch Vernetzung,
Analogiebildung, Herstellung von Assoziationen, Hierarchien,
Ordnungsstrukturen) ist effektiver.

\item
Ein Verstehen von Inhalten als Voraussetzung f\"{u}r das Lernen
ist unter Umst\"{a}nden nur in einer Gesamtschau m\"{o}glich.
\item
Eine Erarbeitung von Inhalten unter verschiedenen Blickwinkeln,
durch verschiedene Quellen, vermittelt durch verschiedene
Personen f\"{o}rdert eine Vernetzung dieser Inhalte.
\end{itemize}

\subsection{Organisatorische Aspekte}

\begin{itemize}
\setlength{\itemsep}{0mm}
\item Schulorganisation
\item
St\"{a}rkere Betonung des Klasslehrerprinzips anstelle
des Fachlehrerprinzips.
\item Stundenplan:
Eine st\"{a}rkere Flexibilit\"{a}t wird erm\"{o}glicht, epochale Phasen
k\"{o}nnen besser eingebracht werden.
\item
Aus- und Weiterbildung der Lehrer.
Neigungen oder Abneigungen der Lehrer.
\end{itemize}

\subsection{Grundhaltung der Physik und Physikdidaktik}

Eher verhalten aus Sorge um die Gewichtigkeit der Physik.
F\"{U}U wird als Chance begriffen, aufbauend auf dem
Fachunterricht zus\"{a}tzlich den obigen Vorteilaspekten zu gen\"{u}gen.

Das Thema F\"{U}U wird in der Physikdidaktik eher von konkreten
Themen her angegangen als von einer Gesamtsicht.

\bip
Vier Thesen zum f\"{a}cher\"{u}bergreifenden Unterricht
(formuliert von Horst Lochhaas, MNU 49/8 (1996)).
\begin{enumerate}
\item
F\"{a}cherverbindende Ziele und Arbeitsformen setzen als
Referenzsystem das Fach voraus.
\item
Der Erwerb fundamentaler naturwissenschaftlicher
Zusammenh\"{a}nge und Denkweisen im Fachunterricht hat Vorrang
gegen\"{u}ber einem f\"{a}cher\"{u}bergreifenden Unterricht um jeden Preis.
\item
F\"{a}cherverbindender Unterricht darf methodisch nicht einengen.
\item
F\"{a}cherverbindender Unterricht ist zeitlich und fachlich m\"{o}glich.
\end{enumerate}


\subsection{M\"{o}gliche Leitmotive}
M\"{o}gliche Leitmotive eines f\"{a}cher\"{u}bergreifenden Unterrichts
\begin{itemize}
\setlength{\itemsep}{0mm}
\item
Philosophische Aspekte.
\item
Historische Entwicklungen.
\item
Gesellschaftliche Relevanz.
\item
Konkreter Bezug von Physik zu einem anderen ausgew\"{a}hlten Fach: Physik und XYZ.
(Mathematik, Chemie, Biologie, Arbeitslehre, Technik, Wirtschaft,
Informatik, Religion, Geographie, Geschichte, Sport, Musik,
Kunst, Deutsch, Sprachen, Psychologie)
\item
Physik, angewandt in anderen Wissenschaftsdisziplinen:
Astronomie, Geologie, Arch\"{a}ologie, Pal\"{a}ontologie.
\item
Physik und Lebensbereiche:
Verkehr, Hausbau (Bauphysik), Handwerk,
Industrie, Haushalt, Freizeit.
\item
Analogien, Symmetrien.
\item
Experimentieren.
\end{itemize}

\subsection{M\"{o}gliche Themenbereiche}
\begin{itemize}
\setlength{\itemsep}{0mm}
\item Der Mensch.
\begin{itemize}
\setlength{\itemsep}{0mm}
\item
Wahrnehmung: Sehen, H\"{o}ren, Sinnesorgane.
\item
Bewegung.
\item
Medizinische Aspekte: Blutkreislauf, Nervensystem.
\end{itemize}
\end{itemize}
\begin{itemize}
\item Wetter und Klima.
\item Umwelt und \"{O}kologie, Zukunft der Welt.
\item Energie und Ressourcen.
\item Das Wasser.
\item Das Fliegen.
\item Fahrzeuge (Fahrrad, Moped, Auto,\dots)
\item Medizintechnik.
\end{itemize}




%------------------------------------------------------------------------------------------------------------------------------------------------------
% EVA-Teil eingefuegt von J Hlawatsch am 08.08.24

\section{Eigenverantwortliches Arbeiten}\label{EVA}

Eigenverantwortliches Lernen und Arbeiten (EVA) im Unterricht ist eine Methode, die darauf abzielt, Sch\"ulern mehr Autonomie und Verantwortung in ihrem Lernprozess zu \"ubertragen. Diese Methode ist eine Antwort auf den soziokulturellen Wandel und die ver\"anderten Lebensbedingungen von Sch\"ulern, die durch Medienkonsum und den Zerfall der Kernfamilie gekennzeichnet sind. Dieser Wandel hat zur Folge, dass die Erfordernisse traditioneller Lehrmethoden hinsichtlich der eigenverantwortlichen Vor- und Nachbereitung von Unterricht nicht mehr erf\"ullt werden und Sch\"uler Unselbstst\"andigkeit und mangelnde Motivation aufweisen.

\subsection{Arbeitsdefinition}
EVA umfasst jede Form des Lehrens und Lernens, bei der Lernende eigenständig Lernhandlungen ausführen oder Entscheidungen über verschiedene Lernvariablen treffen.

Dies beginnt, wenn der Unterricht nicht mehr ausschließlich lehrerzentriert ist, sondern Schüler aktiv in den Lernprozess eingebunden werden. Wichtige Variablen umfassen Lerninhalt, Lerntempo, Lernort, Lernmaterial, Lernorganisation und Lernpartner.

\subsection{Methodische Umsetzung von EVA}
Auswahl nach Eigenst\"andigkeit der Sch\"uler aufsteigend geordnet:
\begin{itemize}
	\item \textbf{Audience Response:} Ein System, das Feedback der Sch\"uler einholt und interaktiven Unterricht erm\"oglicht.
	\item  \textbf{Ich-Du-Wir:} Auch bekannt als \glqq Think-Pair-Share\grqq, eine Methode, bei der Sch\"uler zunächst individuell, dann in Paaren und schlie\ss lich im Plenum über ein Thema nachdenken und diskutieren.
	\item \textbf{Planspiel:} Eine handlungsorientierte Methode, bei der Sch\"uler durch Rollenspiele reale Probleme l\"osen.
	\item \textbf{Projektarbeit:} Sch\"uler arbeiten an einem Lernprodukt, oft \"uber mehrere Tage, und verkn\"upfen dabei verschiedene Fachinhalte.
	\item \textbf{Freiarbeit:} Sch"uler haben vollst\"andige Wahlfreiheit \"uber (fast) alle Lernvariablen und arbeiten selbstst\"andig an vorgegebenen Lernzielen.
\end{itemize}

\subsection{Psychologische Begr\"undung}
Die Notwendigkeit von EVA l\"asst sich auch psychologisch begr\"unden. Die Selbstbestimmungstheorie von Deci und Ryan besagt, dass die Befriedigung der Grundbed\"urfnisse nach Autonomie, Kompetenz und sozialer Eingebundenheit wesentlich f\"ur Motivation und Lernprozesse ist. EVA erf\"ullt dies.

\subsection{Verwandte Konzepte}
\begin{itemize}
	\item \textbf{Autonomie-unterst\"utzendes Lehren:} Unterrichtsmehtoden, die das Gef\"uhl von Autonomie bei Sch\"ulern f\"ordern.
	\item \textbf{Selbstgesteuertes Lernen:} Lernende diagnostizieren ihre Lernbed\"urfnisse, formulieren Ziele, identifizieren Ressourcen, w\"ahlen Lernstrategien und evaluieren ihre Lernergebnisse selbstst\"andig.
	\item \textbf{Selbstreguliertes Lernen:} Ein holistisches Modell, das kognitive, metakognitive und motivationale Prozesse umfasst, die den Lernprozess steuern.
	\item \textbf{Selbstorganisiertes Lernen (SOL):} Ein umfassendes Unterrichtskonzept, das Sch\"uleraktivit\"at in den Mittelpunkt stellt und schulorganisatorisch verankert wird.
\end{itemize}

\subsection{Umsetzung im LehrplanPlus}
Der bayerische LehrplanPlus integriert eigenverantwortliches Arbeiten (EVA) im Physikunterricht ab der 11. Klasse. Dabei bearbeiten Sch\"uler selbstst\"andig oder in Gruppen physikalische Themen und erstellen Lernprodukte wie Pr\"asentationen. Typische Abl\"aufe beinhalten:

\begin{itemize}
	\item Vorbesprechung
	\item Recherche und Erstellung der Lernprodukte
	\item Pr\"asentation und Diskussion der Ergebnisse
\end{itemize}

Die Leistungserhebung erfolgt durch Portfolios oder Kurztests, und Reflexionsphasen fördern die Selbstorganisationskompetenz der Sch\"uler. Themenbereiche umfassen:

\begin{itemize}
	\item Astronomische Weltbilder
	\item Spezielle Relativitätstheorie
	\item Energieversorgung
	\item Statische elektrische und magnetische Felder
	\item Elektromagnetische Induktion und Schwingungen
	\item Das Ohr und neuronale Signalleitung
	\item Strukturuntersuchungen zum Aufbau der Materie
	\item Das Sonnensystem und Sterne
\end{itemize}



