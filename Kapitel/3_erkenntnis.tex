\chapter{Die Erkenntnis von Natur -- durch die Physik}\label{Erkenntnis}

{\bf Jeder} Mensch beobachtet unbefangen, unbewusst seine (Um-)Welt, die Natur, den Lebensalltag,
die technische Zivilisation. Er nimmt --- mit Hilfe der Sinnesorgane --- Ph\"{a}nomene wahr.

\begin{uea}
	Notieren Sie m\"{o}glichst viele Ph\"{a}nomene, die ein Schulkind wahrnimmt / kennt / mitvollzieht!
\end{uea}


\section{Physikalischer Erkenntnisweg}

Ein ,,Mensch mit physikalischer Zuneigung'' bem\"{u}ht sich,
\begin{itemize}
	\item diese Ph\"{a}nomene zu sammeln, zu erfassen,
	\item Zusammenh\"{a}nge zwischen diesen Ph\"{a}nomenen herzustellen,
	\item gemeinsame Erkl\"{a}rungen (Ursachen, Gesetze) f\"{u}r verschiedene Ph\"{a}nomene aufzufinden,
	\item die Erkl\"{a}rungen zu ordnen, zu systematisieren \quad und
	\item sie in exakter Form unter Benutzung rational-logischer Kategorien des menschlichen Geistes, oft in mathematischer Sprache, darzustellen.
\end{itemize}

Es entstehen dabei physikalische
\begin{itemize}
	\item Begriffe (Abstand, Zeit, Energie, Kraft, Drehimpuls)
	\item Modelle (el.\ Strom, Atommodell, Teilchenmodell)
	\item Teilgebiete (Mechanik, Optik, Elektrizit\"{a}tslehre, Optik, Atomphysik)
	\item Theorien (Newton'sche Mechanik, Maxwell'sche Elektrodynamik, Quantenmechanik, QED, GUT,\dots),
\end{itemize}

\mip
Da sich die Ph\"{a}nomene bei unbefangener Beobachtung teilweise sehr uneinheitlich, komplex,
unerkl\"{a}rlich darstellen, stellt der Physiker im Experiment gezielte Fragen an die Welt (Natur),
er achtet dabei auf
\begin{itemize}
	\item Beseitigung st\"{o}render Einfl\"{u}sse (Reibung, Ersch\"{u}tterungen, W\"{a}rmeverlust,\dots )
	\item Nachvollziehbarkeit (mit anderen Apparaturen, an beliebig anderem Ort, von anderen Personen)
	\item Wiederholbarkeit (zu beliebiger Zeit)
	\item Quantitative Erfassung (Messprozess)
	\item Genaue Dokumentation.
\end{itemize}

Es resultiert ein Wechselspiel aus
\begin{itemize}
	\item Experimentalphysik (empirische Methode) \quad und
	\item Theoretischer Physik (induktive und axiomatisch-deduktive Schlussfolgerungen).
\end{itemize}

Dieses Wechselspiel besteht in einer Abfolge von
\begin{itemize}
	\item Hypothesenbildung: Man gelangt zu Vermutungen \"{u}ber die Wirklichkeit durch intuitives --- deduktives Schlie{\ss}en auf der Grundlage schon bekannter Erkenntnisse.
	\item Verifikation an Beispielen bzw.\ Falsifikation
	\item Deutung innerhalb bestehender Theorien
	\item oder Erweiterung der bestehenden Theorien
\end{itemize}

\bip\bip
\section{Physik und Erziehung}
In welcher Form und mit welcher Intensit\"{a}t k\"{o}nnen Schulkinder an diesem Prozess teilnehmen?

Prinzipiell stehen sie einem Physiker nicht nach: Sie nehmen Ph\"{a}nomene wahr und entwickeln ihre eigenen Erkl\"{a}rungsmodelle. Sie bedienen sich dabei ihrer eigenen
\begin{itemize}
	\item entwicklungspsychologisch-altersgem\"{a}{\ss}en \quad und
	\item durch Lernen aus der Umwelt und sozialem Milieu
\end{itemize}
erworbenen Begriffswelt und Denkmuster.

\mip
Physik in der Schule soll nicht verstanden werden
\begin{itemize}
	\item als ein lediglich im Niveau herabgesetzter Wissensfundus,
	\item sondern als Prozess, an dem grunds\"{a}tzlich jeder Mensch und jedes Kind teilnehmen kann.
\end{itemize}

Das Streben nach physikalischer Erkenntnis ist Bestandteil der menschlichen Natur.

\mip
Die Aufgabe der ,,Physik in der Schule'' ist es, dieses Bestreben geeignet zu begleiten und
zu verfeinern. Dabei ist wichtig:
\begin{itemize}
	\item Kenntnis, wie Kinder ,,Physik vollziehen'',
	\item Kenntnis, wie Physiker ,,Physik vollziehen'',
	\item Kenntnis, wie der ,,Physik-Prozess des Physikers'' an den ,,Physik-Prozess im Kind'' angekoppelt werden kann.
\end{itemize}

Einige Grundthesen zur Physik im Unterricht:

\begin{enumerate}
	\item Der Mensch in allen Facetten seiner Gesamtpers\"{o}nlichkeit steht im Mittelpunkt jeden Unterrichts. So sind die Ehrfurcht vor dem Leben, die Achtung der Menschenrechte und das Bem\"{u}hen um eine Menschen-Bildung st\"{a}ndig neu zu verwirklichende Prinzipien in jeder Begegnung von Lehrer und Sch\"{u}ler. (Albert Schweitzer, Tausch/Tausch: Humanistische Pers\"{o}nlichkeitspsychologie)
	\item Die Physik umgibt heute eine Aura des Technizismus (vgl.\ das Negativ-Image infolge der Kernkraft- und Atombombendiskussion). Physik ist aber auch --- wenn nicht: vor allem --- eine Kulturleistung der Menschheit. (Vgl.\ die Hochachtung vor Nobelpreistr\"{a}gern).
	\item Physik als Wissen um Fakten und Methoden ist unverzichtbar in der Bew\"{a}ltigung des Lebens in unserer Gesellschaft (Industrie und Technik). Gleichwohl ist Physik \emph{nur ein} Bestandteil unseres Lebens.
	\item Charisma und Wissen um Didaktik sind zwei wesentliche Bestimmungsst\"{u}cke des Lehrerverhaltens.
	\item Beuys: Jeder Mensch ist ein K\"{u}nstler. Wagenschein (sinngem\"{a}{\ss}): Jeder Mensch ist ein Physiker: Aufgabe des Lehrers ist es, den Prozess der Physik im Sch\"{u}ler zu wecken und zu stimulieren.
\end{enumerate}
