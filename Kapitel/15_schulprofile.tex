\chapter{Schulprofile}

\section{Das Profil des  Gymnasiums}\label{Gymnasium} 

Das Gymnasium ist eine weiterf\"{u}hrende Schulform in Deutschland, die auf eine umfassende, wissenschaftsorientierte Allgemeinbildung abzielt und mit dem Abitur abgeschlossen wird. Das Abitur berechtigt zum Studium an Universit\"{a}ten und Hochschulen und ist der h\"{o}chste allgemeinbildende Schulabschluss in Deutschland.

Wesentliche Merkmale des Gymnasiums sind:

\begin{itemize}
	\item \textbf{Bildungsangebot:} Das Gymnasium bietet eine breite und vertiefte Allgemeinbildung in einer Vielzahl von F\"{a}chern. Dazu geh\"{o}ren neben den Kernf\"{a}chern Deutsch, Mathematik und Fremdsprachen auch Naturwissenschaften, Gesellschaftswissenschaften, Kunst, Musik und Sport. Der Unterricht ist darauf ausgerichtet, die Sch\"{u}ler zu selbstst\"{a}ndigem Denken und wissenschaftlichem Arbeiten zu bef\"{a}higen.
	
	\item \textbf{Fremdsprachenunterricht:} Ein besonderes Merkmal des Gymnasiums ist der intensive Fremdsprachenunterricht. In der Regel beginnen die Sch\"{u}ler mit einer ersten Fremdsprache ab der 5. Klasse und einer zweiten Fremdsprache ab der 6. oder 7. Klasse. Oft besteht die M\"{o}glichkeit, eine dritte Fremdsprache oder ein bilinguales Angebot zu w\"{a}hlen.
	
	\item \textbf{Oberstufe und Abitur:} Die gymnasiale Oberstufe (Klassen 10 bis 12 oder 11 bis 13, je nach Bundesland) ist modular aufgebaut und erm\"{o}glicht den Sch\"{u}lern eine gewisse F\"{a}cherwahl nach ihren Interessen und Begabungen. Die Sch\"{u}ler bereiten sich auf das Abitur vor, das aus einer Kombination von schriftlichen und m\"{u}ndlichen Pr\"{u}fungen besteht. Das Abitur qualifiziert f\"{u}r das Studium an Universit\"{a}ten und Fachhochschulen.
	
	\item \textbf{Wissenschaftsorientierung:} Der Unterricht am Gymnasium ist stark auf wissenschaftliche Arbeitsweisen ausgerichtet. Dies umfasst analytisches Denken, methodisches Arbeiten und die F\"{a}higkeit zur kritischen Reflexion. Projekte, Facharbeiten und Experimente f\"{o}rdern die forschungsorientierte Herangehensweise an Themen.
	
	\item \textbf{Breite Wahlm\"{o}glichkeiten:} Das Gymnasium bietet eine Vielzahl von Wahlpflichtf\"{a}chern und Profilen an, die es den Sch\"{u}lern erm\"{o}glichen, Schwerpunkte nach ihren Interessen zu setzen. Diese Profile k\"{o}nnen in Bereichen wie Naturwissenschaften, Sprachen, Musik oder Kunst liegen.
	
	\item \textbf{Pers\"{o}nlichkeitsentwicklung und Sozialkompetenz:} Neben der fachlichen Ausbildung legt das Gymnasium Wert auf die Entwicklung sozialer Kompetenzen und der Pers\"{o}nlichkeitsbildung. Durch Klassenfahrten, Sch\"{u}leraustausche, AGs und Projekte werden Teamf\"{a}higkeit, Verantwortungsbewusstsein und Eigeninitiative gef\"{a}rdert.
	
	\item \textbf{Studien- und Berufsorientierung:} Das Gymnasium bereitet die Sch\"{u}ler nicht nur auf das Abitur, sondern auch auf das sp\"{a}tere Studium oder Berufsleben vor. Berufs- und Studienberatungen, Praktika und Informationsveranstaltungen unterst\"{u}tzen die Sch\"{u}ler bei ihrer beruflichen Orientierung.

\end{itemize} 

Das Gymnasium bietet somit eine anspruchsvolle und vielseitige Ausbildung, die auf ein breites Spektrum an akademischen und beruflichen M\"{o}glichkeiten vorbereitet. Es legt den Grundstein f\"{u}r ein weiterf\"{u}hrendes Studium und f\"{o}rdert die umfassende Entwicklung der Sch\"{u}ler in intellektueller, sozialer und pers\"{o}nlicher Hinsicht.


\bip\bip
\section{Das Profil der Realschule}\label{Realschule} 

Die Realschule ist eine weiterf\"{u}hrende Schulform in Deutschland, die nach der Grundschule beginnt und in der Regel nach der 10. Klasse mit einem mittleren Schulabschluss (Realschulabschluss) endet. Sie bietet eine praxisnahe und zugleich theoretisch fundierte Allgemeinbildung, die auf verschiedene Bildungs- und Berufsperspektiven vorbereitet.

Wesentliche Merkmale der Realschule sind:

\begin{itemize}
	
	\item \textbf{Bildungsangebot:} Die Realschule vermittelt eine breite Allgemeinbildung in den Kernf{\ss}chern wie Deutsch, Mathematik und Fremdsprachen sowie in den Naturwissenschaften, Gesellschaftswissenschaften und F\"{a}chern wie Kunst, Musik und Sport. Neben theoretischem Wissen werden auch praktische F\"{a}higkeiten vermittelt, oft in Form von Projekten oder Praxisphasen.
	
	\item \textbf{Berufsorientierung:} Ein zentrales Ziel der Realschule ist die Vorbereitung auf den Einstieg in die Berufswelt. Dies geschieht durch eine enge Verzahnung von Theorie und Praxis, z.B. durch Betriebspraktika, Berufsberatung und Projekte mit Unternehmen. Die Realschule legt gro{\ss}en Wert auf die F\"{o}rderung von Schl\"{u}sselkompetenzen, die f\"{u}r das Berufsleben wichtig sind.
	
	\item \textbf{Differenzierung und F\"{o}rderung:} Die Realschule bietet verschiedene Wahlpflichtf\"{a}cher an, die es den Sch\"{u}lern erm\"{o}glichen, nach ihren Interessen und Begabungen Schwerpunkte zu setzen, z.B. in den Bereichen Wirtschaft, Technik oder Sprachen. Zudem gibt es gezielte F\"{o}rderangebote f\"{u}r Sch\"{u}ler, die Unterst\"{u}tzung in bestimmten F\"{a}chern ben\"{o}tigen.
	
	\item \textbf{Abschluss und Perspektiven:} Der Abschluss an der Realschule qualifiziert f\"{u}r den Eintritt in eine duale Ausbildung oder in weiterf\"{u}hrende Schulen wie Fachoberschulen, Berufskollegs oder das Gymnasium (unter bestimmten Voraussetzungen). Der Realschulabschluss ist anerkannt und bietet eine solide Grundlage f\"{u}r vielf\"{a}ltige berufliche und akademische Wege.
	
	\item \textbf{Soziale und Pers\"{o}nlichkeitsentwicklung:} Neben der fachlichen Ausbildung f\"{o}rdert die Realschule die soziale Kompetenz und Pers\"{o}nlichkeitsentwicklung der Sch\"{u}ler. Durch Klassenfahrten, Projekte und au{\ss}erunterrichtliche Aktivit\"{a}ten werden Teamf\"{a}higkeit, Verantwortungsbewusstsein und Selbstst\"{a}ndigkeit gest\"{a}rkt.
	
	Die Realschule bietet somit eine ausgewogene Mischung aus praxisorientierter Bildung und theoretischem Wissen, die sowohl f\"{u}r den direkten Einstieg in die Berufswelt als auch f\"{u}r weiterf\"{u}hrende Bildungswege geeignet ist.
\end{itemize}

%\section{Das Profil der Mittelschule}\label{Mittelschule} 
%Die Mittelschule ist
%eine weiterf\"{u}hrende Schule, sie umfasst die Jahrgangsstufen
%5 -- 9 (10).
%
%Grunds\"{a}tzliches Spannungsfeld:
%\begin{quote}
%	Wissenschaftsorientierung   \q $\ctr$\q
%	Person-Welt-Orientierung
%\end{quote}
%
%\mip
%\"{U}bergeordnete Erziehungsziele sind (Vgl.\ Blatt 1):
%
%\begin{itemize}
%	\item
%	Grundlegende Allgemeinbildung: Kenntnisse, Ganzheitliche Weltsicht.
%	\item
%	Verantwortung in der Welt: Kultur, Werte, politische Bildung.
%	\item
%	Lebensbew\"{a}ltigung: Ern\"{a}hrung, Pers\"{o}nliche Entwicklung,
%	Freizeit, Medien, Verkehr.
%	\item
%	Orientierung f\"{u}r das Arbeitsleben: Kenntnisse f\"{u}r das
%	Berufsleben, Hilfe bei der Berufswahl.
%\end{itemize}
%
%Folgende Abschl\"{u}sse sind m\"{o}glich:
%\begin{itemize}
%	\item
%	Erfolgreicher Hauptschulabschluss
%	($\to$ Ausbildung in Handwerk, Industrie, Dienstleistung)
%	\item
%	Qualifizierender Hauptschulabschluss
%	($\to$ Meister, Techniker, mittlerer nichttechnischer
%	Verwaltungsdienst)
%	\item
%	Mittlerer Schulabschluss ($\to$ FOS, BAS, Studium)
%\end{itemize}
%
%----------
%----------
%\mip
%\pph{\"{A}u{\ss}ere Form des Lehrplans} VERALTET!
%
%Die drei Ebenen:
%\begin{itemize}
%	\item
%	Grundlagen und Leitlinien.
%	\item
%	\"{U}bergeordnete Unterrichts und Erziehungsaufgaben
%	\begin{quote}
%		Fachbezogen --- Fach\"{u}bergreifend.
%	\end{quote}
%	\item
%	Einspaltige Fachlehrpl\"{a}ne.
%	\begin{itemize}
%		\item
%		Lernziele (vgl.\ auch S.\ 17) werden jeweils in
%		einer \"{U}berschrift (Zweistellige GliederungsNummer) beschrieben.
%		Die Lernziele sind nach didaktischen Schwerpunkten geordnet (S.\ 17)
%		\item
%		Es folgen die Lerninhalte ((Dreistellige GliederungsNummer)
%		\item
%		Die Einzelinhalte sind unter Spiegelstrichen aufgelistet.
%	\end{itemize}
%	\item
%	Es ist zu Beginn eines Schuljahres ein aktueller
%	Klassenlehrplan zu erstellen mit Ber\"{u}cksichtigung
%	\begin{itemize}
%		\item
%		von Absprachen und Querverweisen,
%		\item
%		des Schulbuchs, der Medienauswahl,
%		\item
%		Ortes der Schule,
%		\item
%		der Jahreszeitlichen Gegebenheiten.
%	\end{itemize}
%
%\end{itemize}
%
%Computereinsatz: S.\ 15, S.\ 50., S.\ 66.
