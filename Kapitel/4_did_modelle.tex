\chapter{Didaktische Modelle}\label{DidMod}

Dem Lernenden sind didaktische Modelle als abstrakte Grafiken aus Boxen und Pfeilen aus dem erziehungswissenschaftlichen Unterricht bekannt, in denen Pfeile von allem auf alles zeigen. In diesem Abschnitt soll die Bedeutung von Didaktischen Modellen als Grundlage der Unterrichtsgestaltung betont werden, denn die Prinzipien der Unterrichtsgestaltung greifen auf solche Modelle nat\"{u}rlich intrinsisch zur\"{u}ck, siehe hierzu sp\"{a}ter Kapitel \ref{Entwurf}.

\mip

Ein allgemeindidaktisches Modell ist nach \textcite{JankMeyer} \begin{quote}{\glqq}\emph{ein erziehungswissenschaftliches Theoriegeb\"{a}ude zur Analyse und Modellierung didaktischen Handelns in schulischen und nichtschulischen Handlungszusammenh\"{a}ngen. \dots [es] stellt den Anspruch, theoretisch umfassend und praktisch folgenreich die Voraussetzungen, M\"{o}glichkeiten, Folgen und Grenzen des Lehrens und Lernens aufzukl\"{a}ren.}{\grqq} \end{quote} Insofern stellt ein didaktisches Modell einen eher formalen Rahmen dar, {\glqq}\dots innerhalb dessen didaktisches Handeln begr\"{u}ndet und strukturiert werden kann.{\grqq} Didaktische Modelle helfen dabei, Unterricht hinsichtlich seiner Ziele und Methoden diskutierbar und beurteilbar zu machen.

\mip

Ein aktuelles, oft zugrunde gelegtes didaktisches Modell, ist das  $\to$\emph{Berliner Modell}, das in den 1960er Jahren von Paul Heimann entwickelt wurde.  Das Berliner Modell bietet Lehrkr\"{a}ften ein klar strukturiertes Schema zur systematischen Planung und Reflexion des Unterrichts, das sowohl didaktische Entscheidungen als auch die spezifischen Lernbedingungen ber\"{u}cksichtigt. Es unterst\"{u}tzt die Lehrkraft dabei, den Unterricht zielgerichtet und den Bed\"{u}rfnissen der Lernenden entsprechend zu gestalten.

\mip

Das Modell basiert auf der Analyse und Gestaltung von sechs zentralen Elementen, die in zwei Kategorien unterteilt sind:

\begin{enumerate}

	\item Entscheidungsfelder

		\begin{itemize}

			\item Ziele: Welche Lernziele sollen erreicht werden? Hier wird festgelegt, welche Kenntnisse, F\"{a}higkeiten oder Haltungen die Sch\"{u}lerinnen und Sch\"{u}ler durch den Unterricht erwerben sollen.

			\item Inhalte: Welche Themen oder Inhalte werden behandelt? Die Auswahl und Anordnung der Unterrichtsinhalte m\"{u}ssen den Lernzielen entsprechen.

			\item Methoden: Welche Lehr- und Lernmethoden werden angewendet? Es wird festgelegt, wie die Inhalte vermittelt werden und welche didaktischen Prinzipien zum Einsatz kommen.

			\item Medien: Welche Materialien und Medien werden verwendet? Es geht um die Auswahl und den Einsatz von Unterrichtsmaterialien, die den Lernprozess unterst\"{u}tzen.

		\end{itemize}



	\item Bedingungsfelder

		\begin{itemize}

			\item	 Anthropogene Voraussetzungen: Welche Eigenschaften und F\"{a}higkeiten bringen die Lernenden mit? Dies umfasst Aspekte wie Vorkenntnisse, Lernmotivation und soziale Hintergr\"{u}nde der Sch\"{u}lerinnen und Sch\"{u}ler.

			\item Soziokulturelle Voraussetzungen: In welchem Kontext findet der Unterricht statt? Dazu z\"{a}hlen schulische Rahmenbedingungen, Klassengr\"{o}{\ss}e, Schulform, gesellschaftliche Erwartungen und Ressourcen.

		\end{itemize}

\end{enumerate}	



\pph{Wichtige Merkmale:}

\begin{itemize}

	\item Interdependenz der Felder: Die Entscheidungsfelder und Bedingungsfelder sind miteinander verkn\"{u}pft und beeinflussen sich gegenseitig. Eine \"{A}nderung in einem Feld kann Auswirkungen auf die anderen haben.

	\item Reflexionsm\"{o}glichkeit: Das Modell betont die Bedeutung der Reflexion und Anpassung des Unterrichts, basierend auf den Erfahrungen und Ergebnissen des Unterrichtsprozesses.

\end{itemize}	

Das $\to$\emph{Hamburger Modell} von Wolfgang Schulz aus den 1970er Jahren ist eine Weiterentwicklung des Berliner Modells. Es baut auf dem Berliner Modell auf, erweitert es aber um eine tiefere Reflexion der sozialen und ethischen Dimensionen des Unterrichts und legt mehr Wert auf die Erziehung und soziale Verantwortung innerhalb der Bildungsprozesse. Das Hamburger Modell



\begin{itemize}

	\item  legt st\"{a}rkeren Wert auf die kritische Reflexion von Lehr-Lern-Prozessen, einschlie{\ss}lich der ethischen und sozialen Implikationen

	\item integriert die Frage nach sozialen und ethischen Werten st\"{a}rker in die Unterrichtsplanung und betont die soziale Interaktion und die gesellschaftliche Verantwortung des Unterrichts

	\item versteht Didaktik umfassender als Erziehungswissenschaft, die sich nicht nur auf den Unterricht, sondern auch auf die Pers\"{o}nlichkeitsentwicklung der Lernenden richtet

\end{itemize}



Wie wendet man ein solches Modell an, bzw. welche Bedeutung haben solche Modelle f\"{u} die t\"{a}gliche Praxis der Lehrpersonen? Zur Beantwortung dieser Frage schaue man sich beispielsweise die Struktur des schriftlichen Unterrichtsentwurfes in Kapitel \ref{Entwurf} einmal genauer an. Die Bedingungs- und Entscheidungsfelder des Berliner Modells werden hier direkt beantwortet. Das Berliner Modell bildet somit das Fundament bei der Unterrichtsgestaltung.





