\chapter{Literatur zur Physik und Didaktik}

\section{Empfohlene Literatur zur Vorlesung {\glqq}Grundlagen der Fachdidaktik{\grqq} der UBT}
\begin{itemize}
	\item \fullcite{KircherGirwidzHaussler3}
	\item \fullcite{LabuddeMetzger}
	\item \fullcite{Schecker}
	\item \fullcite{Wilhelm}
	\item \fullcite{JankMeyer}
	\item \fullcite{KircherSchneider}
	\item \fullcite{Boxer}
	\item \fullcite{JPMeyn}
\end{itemize}


\section{Zeitschriften}
\begin{itemize}
	\item %\fullcite{PhiuZ}
	Physik in unserer Zeit
	\item %\fullcite{PhuD}
	Physik und Didaktik
	\item %\fullcite{PdNPh}
	Praxis der Naturwissenschaften -- Physik
	\item %\fullcite{NiUPh}
	Naturwissenschaften im Unterricht -- Physik
	\item %\fullcite{PhBl}
	Physikalische Blätter
	\item %\fullcite{MNU}
	Der Mathematisch-Naturwissenschaftliche Unterricht
	\item %\fullcite{SpdW}
	Spektrum der Wissenschaft
\end{itemize}

% das hier waren alles uralte Schulbücher
%\section{Physik inhaltlich, Niveau Sekundarstufe I}
%\begin{itemize}
%	\item \fullcite{BornHubscherLochhaas}
%
%	\item \fullcite{DuitFries}
%	\item \fullcite{WiesnerOptikI}
%\end{itemize}



%Preiswerte Sammlung \"{u}ber alles Wissenswerte
%aus den Naturwissenschaften: \fullcite{MeyerSchmidt}
%
%\section{Physik inhaltlich, Niveau Sekundarstufe II}
%\fullcite{SahmWiller}
%\fullcite{Schlichting}
%
%Formelsammlung: \fullcite{HammerHammer}
%
%\section{Astronomie, Niveau Sekundarstufe II}
%\fullcite{Lermer}
%\fullcite{BeckmannEpperlein}
%\fullcite{Hasemann}
%\fullcite{Henkel}
%
%\section{Physik inhaltlich, Niveau Grundstudium}
%\fullcite{GerthsenVogel},
%\fullcite{AlonsoFinn}.
%
%Formelsammlung: \fullcite{Stoecker}.

%Lexikon: \fullcite{abcPhysik1}, \fullcite{abcPhysik2}.

\section{Fachdidaktik Physik}
\begin{itemize}
	\item \fullcite{KircherGirwidzHaussler1} (downloadbar unter \url{https://katalog.uni-bayreuth.de:443/TouchPoint/perma.do?q=35%3D%22(OCoLC)1220881218%22+IN+%5B2%5D&v=ubt&b=0&l=de})
	\item \fullcite{KircherGirwidzHaussler2} (downloadbar unter \url{https://katalog.uni-bayreuth.de:443/TouchPoint/perma.do?q=35%3D%22(OCoLC)1199763053%22+IN+%5B2%5D&v=ubt&b=0&l=de})
	\item \fullcite{KircherGirwidzHaussler3}
	\item \fullcite{KircherSchneider}
	\item \fullcite{DuitHausslerKircher}
	\item \fullcite{BleichrothDahnckeJung}
	\item \fullcite{Braun}
	\item \fullcite{WagenscheinPDP}
	\item \fullcite{LabuddeATP}
	\item \fullcite{LabuddeEWP}
	\item \fullcite{Muckenfuss}
	\item \fullcite{Ploeger}
%	\item \fullcite{WegeinderPhysikdidaktik1}
%	\item \fullcite{WegeinderPhysikdidaktik2}
%	\item \fullcite{WegeinderPhysikdidaktik3}
	\item \fullcite{WegeinderPhysikdidaktik4}
	\item \fullcite{Willer}
\end{itemize}


\section{Erziehungswissenschaften}
\begin{itemize}
	\item \fullcite{MeyerUMT}
	\item \fullcite{MeyerUMP}
	\item \fullcite{Koeck}
	\item \fullcite{Reble1}
	\item \fullcite{Reble2}
\end{itemize}



\section{Experimente --- Unterhaltsame Physik}
\begin{itemize}
	\item \fullcite{Lindenblatt}
	\item \fullcite{Aulas}
	\item \fullcite{Dussler}
%	\item \fullcite{ABD98}
	\item \fullcite{Berge}
	\item \fullcite{MelenkRunge}
	\item \fullcite{ZeierKW}
	\item \fullcite{KnoffHoff1}
	\item \fullcite{KnoffHoff2}
%	\item \fullcite{Wittmann1}
%	\item \fullcite{Wittmann2}
\end{itemize}


%\section{Schulb\"{u}cher Mittelschule}
%Es handelt sich um Schulb\"{u}cher (evtl.\ mit Kopiervorlagen oder
%Lehrerbegleitb\"{a}nden), die auf den Lehrplan (1997/98)
%bezogen sind.
%Als Beispiel sind jeweils nur die PCB-B\"{u}cher der 7.\ Jahrgangsstufe
%angegeben:
%\begin{itemize}
%\setlength{\itemsep}{0mm}
%\item Bayerischer Schulbuch-Verlag:
%       Natur entdecken \fullcite{Schurius7} \fullcite{Schurius7L}
%
%\item Cornelsen-Verlag: Natur und Technik \fullcite{Hiering7}
%\item Klett-Verlag: Urknall \fullcite{Litz7}
%\item Schroedel:
%       Natur plus \fullcite{ScharfSchulz7} \fullcite{ScharfSchulz7M}
%
%\item Westermann: Natur bewusst \fullcite{HausfeldSchulenberg7}
%                        \fullcite{HausfeldSchulenberg7K}
%                        \fullcite{HausfeldSchulenberg7L}
%\end{itemize}
