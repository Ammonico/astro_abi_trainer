\chapter{Literaturempfehlungen zur Vorlesungen}
Hier finden Sie einige Bücher und Veröffentlichungen, die wir Ihnen begleitend zur Vorlesung zur Lektüre empfehlen. Einige sind im Campusnetz zum freien Download verfügbar. Sie sind mit \faFilePdf[refular] gekennzeichnet und verlinkt.

\section{Fachdidaktische Grundlagenbücher}
E. Kircher veröffentlicht seit geraumer Zeit umfangreiche Grundlagenbücher zur Physikdidaktik; die Werke sind nicht vollständig deckungsgleich.
\begin{itemize}
	\item \fullcite{KircherGirwidzHaussler1} \href{https://link.springer.com/book/10.1007/978-3-662-59490-2}{\faFilePdf[regular]}
	\item \fullcite{KircherGirwidzHaussler2} \href{https://link.springer.com/book/10.1007/978-3-662-59496-4}{\faFilePdf[regular]}
	\item \fullcite{KircherGirwidzHaussler3} \href{https://link.springer.com/book/10.1007/978-3-642-01602-8}{\faFilePdf[regular]}
	\item \fullcite{KircherSchneider} \href{https://link.springer.com/book/10.1007/978-3-642-56386-7}{\faFilePdf[regular]}
\end{itemize}

Weitere Werke:
\begin{itemize}
	\item \fullcite{LabuddeMetzger}
	\item \fullcite{Schecker} \href{https://link.springer.com/book/10.1007/978-3-662-57270-2}{\faFilePdf[regular]}
	\item \fullcite{Wilhelm} \href{https://link.springer.com/book/10.1007/978-3-662-63053-2}{\faFilePdf[regular]}
	\item \fullcite{Boxer}
	\item \fullcite{JPMeyn} \href{https://www.degruyter.com/document/doi/10.1524/9783486721249/html}{\faFilePdf[regular]} 
	\item \fullcite{Muckenfuss}
\end{itemize}

\section{Schulpädagogische Werke}
Die Meyer'sche Trilogie bietet einen guten Überblick über Unterrichtsmethoden und -planung.
\begin{itemize}
	\item \fullcite{JankMeyer}
	\item \fullcite{MeyerUMT}
	\item \fullcite{MeyerUMP}
\end{itemize}
Die Hattie-Studie ist eine groß angelegte Metastudie zu Einflussvariablen des Unterrichts.
\begin{itemize}
	\item \fullcite{Hattie}
\end{itemize}

\section{Zeitschriften}
\begin{itemize}
	\item %\fullcite{PhiuZ}
	Physik in unserer Zeit
	\item %\fullcite{PhuD}
	Physik und Didaktik
	\item %\fullcite{PdNPh}
	Praxis der Naturwissenschaften -- Physik
	\item %\fullcite{NiUPh}
	Naturwissenschaften im Unterricht -- Physik
	\item %\fullcite{PhBl}
	Physikalische Blätter
	\item %\fullcite{MNU}
	Der Mathematisch-Naturwissenschaftliche Unterricht
	\item %\fullcite{SpdW}
	Spektrum der Wissenschaft
\end{itemize}

%\section{Fachdidaktik Physik}
%\begin{itemize}
%%	\item \fullcite{DuitHausslerKircher}
%	\item \fullcite{BleichrothDahnckeJung}
%	\item \fullcite{Braun}
%	\item \fullcite{WagenscheinPDP}
%	\item \fullcite{LabuddeATP}
%	\item \fullcite{LabuddeEWP}
%
%	\item \fullcite{Ploeger}
%%	\item \fullcite{WegeinderPhysikdidaktik1}
%%	\item \fullcite{WegeinderPhysikdidaktik2}
%%	\item \fullcite{WegeinderPhysikdidaktik3}
%	\item \fullcite{WegeinderPhysikdidaktik4}
%	\item \fullcite{Willer}
%\end{itemize}


%\section{Erziehungswissenschaften}
%\begin{itemize}
%	\item \fullcite{MeyerUMT}
%	\item \fullcite{MeyerUMP}
%	\item \fullcite{Koeck}
%	\item \fullcite{Reble1}
%	\item \fullcite{Reble2}
%\end{itemize}



%\section{Experimente --- Unterhaltsame Physik}
%\begin{itemize}
%	\item \fullcite{Lindenblatt}
%	\item \fullcite{Aulas}
%	\item \fullcite{Dussler}
%%	\item \fullcite{ABD98}
%	\item \fullcite{Berge}
%	\item \fullcite{MelenkRunge}
%	\item \fullcite{ZeierKW}
%	\item \fullcite{KnoffHoff1}
%	\item \fullcite{KnoffHoff2}
%%	\item \fullcite{Wittmann1}
%%	\item \fullcite{Wittmann2}
%\end{itemize}