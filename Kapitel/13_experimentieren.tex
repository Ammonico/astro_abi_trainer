\chapter{Experimentieren im Physikunterricht}\label{Experiment}

\section{Das Experiment als physikalisch-
                         erkenntnistheoretische Methode}

Die Physik beschreibt
\begin{itemize}
	\item
	in exakter Sprache (Logik, Mathematik, Fachterminologie)
	\item
	die Ph\"{a}nomene, die ,,unsere Welt'' (Kosmos, Natur,
	Alltag, Technik) grundlegend pr\"{a}gen.
\end{itemize}

Die Beschreibung ist nicht einfach enzyklop\"{a}disch-dokumentierend,
es wird vielmehr ein Gesamtsystem hergestellt, in dem
\begin{itemize}
	\item
	Ordnungsstrukturen und Hierarchien,
	\item
	Analogien und Modellvorstellungen,
	\item
	Ursachen und Wirkungen (Kausalit\"{a}t),
	\item
	Idealisierungen und Anwendungen
\end{itemize}
aufgezeigt werden.

Diese Ph\"{a}nomene nehmen Menschen mittels der Sinnesorgane wahr
\begin{itemize}
	\item
	zum Teil unmittelbar,
	\item
	zum Teil durch bewusste gezielte Herstellung von bestimmten
	Rahmenbedingungen im Experiment.
\end{itemize}

Der Erkenntnisgewinn der Physik vollzieht sich dabei in einem
Wechselspiel von
\begin{itemize}
	\item
	Experimentalphysik: Sie hat die Aufgabe, auf
	empirisch-induktivem Wege ausgehend von den Ph\"{a}nomenen
	Gesetzm\"{a}{\ss}igkeiten zu erschlie{\ss}en \\ und
	\item
	Theoretischer Physik:
	Hier werden auf logisch-deduktivem Wege die Gesetzm\"{a}{\ss}igkeiten
	auf ihre Geschlossenheit hin \"{u}berpr\"{u}ft, zusammengef\"{u}hrt und
	Folgerungen (f\"{u}r die Wirklichkeit) gezogen.
\end{itemize}

Die Physik ist also eine exakte empirische Naturwissenschaft.
Das Experimentieren ist unmittelbarer Bestandteil ihres Wesens.

\subsection{Wesensmerkmale von Experimenten}

Experimente m\"{u}ssen bei Durchf\"{u}hrung
\begin{itemize}
	\item
	zu anderen Zeiten (Wiederholung)
	\item
	an anderen Orten, bei anderen Ausrichtungen,
	\item
	bei Benutzung anderer Materialien, Ger\"{a}te,
	\item
	unter der Regie anderer Experimentatoren
\end{itemize}
zu identischen Ergebnissen f\"{u}hren
(wenn nicht diese Parameter unmittelbare Bestandteile der
Experimentieranordnung sind).

\bip

Experimentieren im Unterricht er\"{o}ffnet also zun\"{a}chst die
Einsicht in eine grundlegende Arbeitsweise der Physik und
damit der Naturwissenschaften \"{u}berhaupt.


\bip\bip
\section{Das Experiment als unterrichtlich-lerntheoretische
                                          Methode}

Unabh\"{a}ngig von der erkenntnistheoretischen Funktion dient das
Experimentieren im Physikunterricht
--- je nach konkreter Umsetzung ---
einer ganzen Reihe von \"{u}bergeordneten Lernzielen:

\begin{itemize}
	\item
	Schulung der Sinnesorgane (Beobachten,\dots)
	\item
	Artikulation der Beobachtungsergebnisse (Protokollierung,
	Sprachliche Beschreibung),
	\item
	Kognitive Entwicklung (Kausales Denken, Abstraktionsf\"{a}higkeit,
	                                Objektivit\"{a}t,\dots)
	\item
	Ausdauer, Geduld, Sorgfalt,
	\item
	Handwerklich-technische Kenntnisse
	(Werkstoffe, Ger\"{a}te, Sicherheitsvorkehrungen,\dots)
	\item
	Handwerklich-technische (Finger-)Fertigkeiten
	(Schrauben, L\"{o}ten, Regeln,\dots)
	\item
	Soziale Entwicklung (besonders bei Gruppenexperimenten).
	\item
	Affektive Dimension: Interesse, Abbau von \"{A}ngsten,
	Aha-  oder Erfolgserlebnisse.
\end{itemize}

oder Unterrichtsprinzipien

\begin{itemize}
	\item
	Handlungsorientierung, Selbstt\"{a}tigkeit (Sch\"{u}leraktivit\"{a}t).
	\item
	Unmittelbarkeit ,,Mit den eigenen Augen sehen!''
	\item
	Anschauung: Alle Sinne werden angesprochen.
	\item
	Alltagsn\"{a}he
	\item
	Wissenschaftsorientierung
\end{itemize}

\bip\bip
\section{Klassifikation von Unterrichtsexperimenten}
Experimente (synonym: Versuche) im Unterricht k\"{o}nnen
klassifiziert werden hinsichtlich verschiedenster Kategorien.

\subsection{Erkenntnistheoretische Funktion}
\begin{itemize}
	\item
	Erarbeitungsexperiment: Induktiv (Ph\"{a}nomen $\to$ Gesetz)
	\item
	Best\"{a}tigungsexperiment: Deduktive Methode (Gesetz $\to$ Ph\"{a}nomen)
	\item
	Gedankenexperiment: Ist das Gesetz in sich schl\"{u}ssig?
	\item
	Messung einer Natur-, Material- oder Ger\"{a}tekonstante.
	\item
	Modellexperiment: Um einen experimentell nicht oder nur schwer
	zug\"{a}nglichen Sachverhalt besser verstehen zu k\"{o}nnen,
	wird das Experiment an einem Modell durchgef\"{u}hrt:
	Bsp.: Maxwell'sche Geschwindigkeitsverteilung.
	\item
	Simulationsexperiment: (Die Wirklichkeit wird vorget\"{a}uscht,
	im Mittelpunkt stehen die Beobachtung und Auswertung).
	\item
	Historisches Experiment.
	\item
	Spielexperiment: Leistung meines K\"{o}rpers beim Treppensteigen.
	Astronomie-Modellspiele, Der Schwerpunkt beim Hochsprung,\dots
\end{itemize}

\subsection{Zeitliche Einordnung in einer Unterrichtseinheit}

\begin{itemize}
	\item
	Motivation oder Einstieg: Weckung eines kognitiven Konflikts ($\to$ Schülervorstellungen, \cref{Schuelervorstellungen})
	oder Show-Effekt.
	\item
	Problemstellung: Ein Experiment wirft ein Problem auf.
	\item
	L\"{o}sung: Erarbeitung (einer Gesetzm\"{a}{\ss}igkeit),
	Klassische Funktion des Experiments.
	
	\item
	Wiederholung, Festigung: Das Experiment wird wiederholt.
	
	\item
	\"{U}bertragung, Integration: Das Experiment wird --- unter ver\"{a}nderten
	Bedingungen bzw.\ Fragestellungen durchgef\"{u}hrt.
%	\item
%	\"{U}bung: %?!
\end{itemize}

\subsection{Intensit\"{a}t und Art der Auswertung}
\begin{itemize}
	\item
	Alternativ oder
	Qualitativ oder Quantitativ (Vgl.\ Elementarisierung).
	\item
	Messreihe, Diagramm.
	\item
	Anzeige, analog-digital, Computererfassung und Weiterverarbeitung.
\end{itemize}

\subsection{Art der Repr\"{a}sentation}
\begin{itemize}
	\item
	Reine Beobachtung eines Ph\"{a}nomens.
	\item
	Freihandversuch.
	\item
	Durchsichtige/Einsichtige Anordnung.
	\item
	Hoher Ger\"{a}teAufwand (Begriff der BlackBox).
	\item
	Computereinsatz.
\end{itemize}

\subsection{Experimentierort}
Physiksaal --- Klassenzimmer --- Pausenhof --- Sporthalle
 --- Sch\"{u}ler-Zuhause --- Natur/Umwelt (Wandertag, Klassenfahrt) --- vorbereitet als Video --- Simulationsexperimentierumgebungen.

\subsection{Experimentator}
\begin{itemize}
	\bitem{Lehrer-Demonstrationsexperiment}
	Die Gr\"{u}nde daf\"{u}r sind vor allem pragmatischer Natur:
	
	\begin{itemize}
		\item Mangelnde Sicherheit f\"{u}r die Sch\"{u}ler oder die Ger\"{a}te
		\item Spontaneit\"{a}t (\"{U}berraschung) beispielsweise bei Freihandversuch oder innerhalb der
		Motivationsphase.
		\item Einheitliche Darbietung f\"{u}r alle Sch\"{u}ler
		(z.B.\ vor Leistungserhebung),
		\item Didaktisch oder fachlich versierte Darbietung durch den ,,Experten''
		\item Sch\"{u}ler\"{u}berforderung  hinsichtlich Kenntnissen, Eingew\"{o}hnung,
		Komplexit\"{a}t, Fertigkeiten
		\item Zeitknappheit ($\gets$ Lehrplan)
		\item Zu geringe Ger\"{a}te- bzw.\ Arbeitsplatzausstattung
		\item Disziplinprobleme
	\end{itemize}

	Lehrerdemonstrationsexperimente sind Bestandteil des eher
	darbietenden Unterrichts.
	
	\bitem{Sch\"{u}ler-Demonstrationsexperiment}
	Z.B.\ im Rahmen eines Sch\"{u}lerreferats.
	
	\bitem{	Sch\"{u}ler-Einzelexperiment}
	Schwierig ist die (ungelernte) spontane Beherrschung simultaner
	Arbeitsabl\"{a}ufe (Einstellen, in Gang setzen, ablesen,\dots),
	\bitem{
	Sch\"{u}ler-Gemeinschaftsexperiment}
	Mischform, beispielsweise im Sch\"{u}lerkreis, Projektunterricht,
	soziale Erfahrungen (Lehrer und Sch\"{u}ler handeln gemeinsam).
	\bitem{
	Sch\"{u}ler-Gruppenexperiment} Siehe unten!
	\bitem{
	Dritte:} im Film, Besichtigung, \dots
\end{itemize}

\subsection{Sch\"{u}lerexperimente in Gruppenarbeit}

Vergleiche auch dazu die \"{U}berlegungen zum Begriff der
Gruppenarbeit als Sozialform.

\begin{itemize}
	\bitem{
	Gruppengr\"{o}{\ss}e} 2 -- 6 Personen, mit oder ohne Lehrer.
	
	\bitem{
	Rahmen} Einfache Experimente, evtl.\ Freihandexperimente,
	einfaches sicheres Ger\"{a}t und Material.
	\bitem{
	Arbeitsauftrag}
	\begin{itemize}
		\item
		Parallel identisch (arbeitsgleich).
		Es k\"{o}nnen beispielsweise Mittelwerte gebildet werden.
		Der Lehrer kann gut steuern, beispielsweise bei typischen Fehlern.
		\item
		Parallel abwechselnd:
		Beispielsweise bei Ger\"{a}te- oder Arbeitsplatzmangel
		\item
		Parallel erg\"{a}nzend (arbeitsteilig):
		Die Sch\"{u}ler f\"{u}hren Experimente zum gleichen Themenkreis mit
		gleichen oder verschiedenen Auftr\"{a}gen durch;
		Ergebnisse f\"{u}hren zu einer gemeinsamen Probleml\"{o}sung
		(Richtung Projektarbeit).
		\item
		Frei (Experimentalfreiarbeit):
		Eine sehr reizvolle (Wagenschein'sche) Idealform, hoch motivierend.
		Kann z.B.\ in Neigungsgruppen, Helfergruppen oder
		Freizeitgruppen organisiert werden.
	
	\end{itemize}
	
	\item
	Eine gewisse \textbf{Steuerung} ist im Allgemeinen notwendig.
	Sie wird beispielsweise durch ein Arbeitsblatt oder eine
	aufgelegte Folie gew\"{a}hrleistet.
	Ein umfangreiches Programm kann so schneller absolviert werden.
	
	
	\begin{itemize}
		\item
		Der Lehrer oder Betreuer wird entlastet, er kann sich
		einzelnen Gruppen zuwenden.
		\item
		Themenstellung, evtl.\ Datum, Name, Klasse.
		\item
		Die Versuchsanordnung (ikonisch oder symbolisch)
		mit Beschriftungen.
		\item
		Die Art der Parameter, wie werden sie eingestellt?
		\item
		Mehr oder weniger genaue (kleinschrittige) Arbeitsanweisungen
		f\"{u}r die durchzuf\"{u}hrenden Versuche.
		\item
		Tabellen oder Koordinatensysteme f\"{u}r die Auswertung.
		\item
		L\"{u}ckentexte f\"{u}r die Ergebnisfixierung.
	
	\end{itemize}
	
	\bitem{
	St\"{a}ndig wiederkehrende T\"{a}tigkeiten} k\"{o}nnen separat einge\"{u}bt werden.
	Umgang mit Schaltungen allgemein, Befestigungstechnik,
	Bedienung von Messger\"{a}ten oder Netzger\"{a}ten, Bunsenbrenner.
	Im Laufe der Zeit sollten Sch\"{u}ler zunehmend selbst\"{a}ndig werden.
	
	\bitem{
	Vorbereitende T\"{a}tigkeiten} (Abzwicken von Dr\"{a}hten,
	Knoten von Schn\"{u}ren o.\"{a}., Entwirren von Kabelgeflechten)
	halten auf, sind aber u.U.\ auch f\"{u}r sich lehrreich.
	\bitem{
	Materialien} (m\"{o}glichst abgez\"{a}hlt, evtl.\ abgeteilt in Beh\"{a}ltern)
	werden bereitgestellt
	(Auf das Mitbringen von Gegenst\"{a}nden von Zuhause kann man sich u.U.\
	nicht verlassen).
	\item
	Innerhalb einer Gruppe k\"{o}nnen \textbf{Funktionen} vereinbart werden:
	Materialholer, Durchf\"{u}hrer, Protokollf\"{u}hrer, Berichterstatter.
	\bitem{Arbeitsplatz} sollte von \"{u}berfl\"{u}ssigen Gegenst\"{a}nden (B\"{u}chern,
	Schultaschen, Essen, Getr\"{a}nken) frei sein.
	Es gen\"{u}gen im Allgemeinen die Versuchsmaterialien und Schreibzeug.
\end{itemize}
	
Besonderheiten in der Stunde:
\begin{itemize}
	\item
	Das Experimentieren in Gruppenarbeit hat seinen Platz innerhalb der
	Erarbeitungsphasen der einzelnen Artikulationsschemata.
	Andere Phasen (Einstieg, Problemfrage, Fixierung,\dots)
	erfolgen gegebenenfalls im Klassenverband.
	\item
	Der Wechsel zwischen den Experimentiert\"{a}tigkeiten und
	Klassengespr\"{a}ch (Arbeitsauftr\"{a}ge, Erkl\"{a}rungen, Korrekturen)
	sollte eigentlich vermieden werden,
	er muss aber einge\"{u}bt werden.
\end{itemize}

\bip\bip
\section{Allgemeine Hinweise}
\begin{itemize}

	\item
	Versuchsanordnung
	\begin{itemize}
		\item
		Bewegungsabl\"{a}ufe, Input-Output-Vorg\"{a}nge, Lichtstrahlen
		sollten von links nach rechts bzw.\ von oben nach unten
		gerichtet sein.
		
		\item
		Die Versuchsanordnung sollte \"{u}bersichtlich sein.
		Dazu dienen zus\"{a}tzliche Markierungen (Plus- bzw.\ Minuspol),
		farbige Aufkleber, farbige Experimentierleitungen.
		\item
		Messinstrumente sollten gut ablesbar sein.
		W\"{a}hle die Messbereiche einsichtig und zweckm\"{a}{\ss}ig!
	\end{itemize}
	
	\item
	Das Problem der Redlichkeit (,,Tricksen''):
	Der Versuchsaufbau oder die Durchf\"{u}hrung werden so
	manipuliert, dass im Nachhinein ein erw\"{u}nschtes Ergebnis
	eintritt (,,Bilderbuchmesswerte'').
	
	Ist das guter Physikuntericht?
	
	\begin{beisp2}
	\begin{itemize}
		\item
		Die Fahrbahn wird geneigt, damit die Reibungskraft
		kompensiert wird.
		\item
		Der Gartenerde oder dem Leitungswasser wird Salz beigemischt,
		damit sie/es sich als leitf\"{a}hig herausstellt.
		\item
		Dem Wasser im Hoffmann'schen Zersetzungsapparat wird Salzs\"{a}ure
		beigegeben, damit die Elektrolyse deutlich eintritt.
		\item
		Der Nullpunktsschieber an einem Kraftmesser wird verstellt.
	\end{itemize}
	\end{beisp2}
	
	\item
	Protokollierung eines Experiments
	(auf Arbeitsblatt, im Hefteintrag,\dots)
	\begin{itemize}
		\item
		Aufbau: Skizze, Schaltbild mit
		Beschriftung oder Texterl\"{a}uterung.
		\item
		Durchf\"{u}hrung:
		Welche Gr\"{o}{\ss}en werden vorgegeben, eingestellt, variiert?
		\item
		Beobachtung.
		Welche Gr\"{o}{\ss}en werden beobachtet bzw.\ gemessen?
		\item
		Erarbeitung des Ergebnisses: Abgeleitete Gr\"{o}{\ss}en,
		graphische Auftragung, G\"{u}ltigkeitsbereich.
		\item
		Deutung, induktiv gewonnenes Gesetz.
	\end{itemize}
	
	\item
	Sicherheit beim Experimentieren: %Siehe EXP! ??
\end{itemize}

\bip\bip
\section{Freihandexperimente}

Vor einigen Jahren noch wurde das eher bel\"{a}chelt. Gegenw\"{a}rtig erfreuen sie
sich als ,,urspr\"{u}ngliche Erfahrung von Natur und Technik'' einer rasanten
Beliebheitssteigerung. %:
%\begin{itemize}
%	\item Grundschule: Genetischer Sachunterricht.
%	\item Mittelschule: Schon immer stark der handlungsorientierte, lebensnah-praktische
%	Ansatz.
%	\item Gymnasium: Natur und Technik-Unterricht in der 5.\ Klasse:
%	\item Realschule: Am st\"{a}rksten eher traditionell.
%\end{itemize}
\mip
Begleitmaterialien aller Art.
\mip
\"{O}ffentlichkeit: Sachb\"{u}cher, selber experimentieren, Fernsehsendungen.
\mip
Es widerspricht geradezu diesem Begriff, wenn man ihn
mit hoher Genauigkeit definieren wollte.
\mip
Die Wortbestandteile ,,Frei'' und ,,Hand'' geben erste Anhaltspunkte.


Im folgenden wird der Versuch unternommen, anhand von
verschiedenen Gesichtspunkten den Begriff n\"{a}her einzugrenzen.

Allgemein wird man von einem Freihandexperiment um so eher sprechen
k\"{o}nnen, je mehr die folgenden Kriterien erf\"{u}llt sind.
\begin{itemize}
	\bitem
	{Nicht-Standardisierung}
	FHEe sollten frei gestaltet sein, vielleicht auf eigenen Ideen
	des Experimentators beruhen.
	Der Einsatz vorgegebener Versuchsaufbauten, wie sie
	beispielsweise von Lehrmittelfirmen angeboten werden,
	widerspricht grunds\"{a}tzlich der Idee von FHEen.
	\bitem
	{Kurze Zeitdauer}
	Die Durchf\"{u}hrung eines FHEs sollte nicht sehr viel Zeit
	(d.h.\ Sekunden bis wenige Minuten) in Anspruch nehmen.
	Sie zeigen {\it Ph\"{a}nomene} eher pr\"{a}gnant-qualitativ als
	umfassend-quantitativ.
	\mip
	Eine individuelle, eigene, evtl.\ gar k\"{u}nstlerische, Gestaltung
	geben dem Unterricht eine ganz andere Intensit\"{a}t.
	
	\bitem
	{Gegenst\"{a}nde und Materialien}
	Sie sollten dem unmittelbaren Erfahrungsbereich der
	Sch\"{u}ler entstammen:
	\begin{quote}
		Spielzeug ($\to$ ,,Spielzeug-Physik''), Haushalt, Hobby,
		Basteln, Freizeit, Werkzeug.
	\end{quote}
	Auch aus der Physiksammlung k\"{o}nnen unter Umst\"{a}nden
	Alltagsgegenst\"{a}nde bereitgestellt werden wie
	\begin{quote}
		Kompass, Lupe, Magnete, Taschenlampe, Battereien, Kleinspannungsnetzger\"{a}t.
	\end{quote}
	
	\mip
	Es erfolgt im Allgemeinen kein oder ein sehr
	einfacher Versuchsaufbau.
	Man kann nicht definitiv ausschlie{\ss}en, dass
	Stativmaterial verwendet wird, eventuell kann man
	aber auch Befestigungen mit W\"{a}scheklammern,
	Buchbeschwerungen, Schraubstock oder \"{a}hnlichem w\"{a}hlen.
	
	\mip
	Kosten sollten eher gering sein (Engl.: Low budget experiments).
	
	\bitem
	{Spontaneit\"{a}t} Mit der ,,freien Hand'' k\"{o}nnte bedeuten,
	dass ein FHE spontan, d.h.\ ohne jede Vorbereitung aus dem
	Unterrichtsgeschehen heraus durchgef\"{u}hrt werden.
	Dies setzt eine gut geordnete, vielleicht reichlich mit
	FHE-Gegenst\"{a}nden ausgestattete Physiksammlung voraus,
	die nat\"{u}rlich auch gepflegt werden muss.
	
	\mip
	Unter Umst\"{a}nden kann ein FHE auch umfangreiche Vorbereitung
	erfordern, die dann in der Bereitstellung von
	Material
	(Bsp.: Eisblock, Selbst gebauter Tripelspiegel,\dots)
	oder in der Ein\"{u}bung des ,,Hand''elns
	(Bsp.: Besenstielgleichgewicht) oder Austestung besteht.
	
	\bitem
	{Bestimmte Unterrichtsprinzipien} werden bedient:
	\begin{itemize}
		\item
		Lebensn\"{a}he (Alltagsn\"{a}he).
		\item
		Handlungsorientierung (Selbstt\"{a}tigkeit, Sch\"{u}leraktivit\"{a}t).
		\item
		Anschauung.
		\item
		\"{A}sthetik.
		\end{itemize}
		
		\item
		{\bf Bestimmte Lernziele} r\"{u}cken in den Mittelpunkt
		\begin{itemize}
		\item
		Affektive Lernziele: Interesse, Freude, Begeisterung.
		\item
		Psychomotorische Lernziele (bei Sch\"{u}lerexperimenten):
		Hand-Fertigkeiten (Basteln, Bedienung einfacher Ger\"{a}te)
		\item
		Kognitive Lernziele: Kenntnis einfacher Ph\"{a}nomene (Wagenschein)
	\end{itemize}
\end{itemize}

\subsection{Andere Gesichtspunkte}
\begin{itemize}

	\bitem
	{Unterrichtsphasen}
	\begin{itemize}
		\item
		Motivation:
		Kognitiver Konflikt (vgl.\ auch N\"{a}he zur Zauberei, Magie),
		Show-Effekt.
		\item
		Erarbeitung:
		Entscheidung: Schwimmt das Plastillinschiff? /
		Kommt die Zwirnrolle? \\
		Vergleich: Gewichstskraft von Styroporblock und Bleikugel \\
		Absch\"{a}tzung: Wie gro{\ss} ist die Schallgeschwindigkeit?
		(Startklappe beim Hundertmeterlauf)
		\item
		\"{U}bung, Festigung, Best\"{a}tigung:
		(Gilt bei einer optischen Abbildung
		$g = 2f$, so ist $B=G$ und $b = g = 2f$.)
		
		Vorzeigen eines (vielleicht zerlegten) technischen Ger\"{a}tes
		Trockenbatterie, Fahrraddynamo, Gl\"{u}hbirne,\dots)
		
		\item
		Hausaufgabe (Strohhalmtr\"{o}te).
	\end{itemize}
	
	\bitem
	{Organisationsformen}
	Vertretungsstunden, Freiarbeit, F\"{a}cher\"{u}bergreifender Unterricht,
	Projektunterricht (Projekttag),
	au{\ss}erhalb der Schule (Kindergeburtstag, Zauberabend).
	
	\bitem
	{Ort des Experimentierens}
	Da FHEe im Allgemeinen nicht gro{\ss}artige Versuchsaufbauten
	erfordern, k\"{o}nnen sie auch leichter an anderen Orten
	als dem Physiksaal durchgef\"{u}hrt werden.
	
	\begin{quote}
		Klassenzimmer, Zuhause, Schulhof, Turnhalle,
		Schwimmbad, Wandertag, Schulfahrt, Schul\-land\-heim.
	\end{quote}
	
	\bitem
	{Sicherheit}
	Nat\"{u}rlich sind auch bei FHEen die
	einschl\"{a}gigen Sicherheitsgrunds\"{a}tze voll zu beachten.
	
	\mip
	Insbesondere bei Sch\"{u}lerexperimenten, die wom\"{o}glich zuhause
	durchgef\"{u}hrt werden, muss gegebenenfalls (per Hefteintrag)
	eine Ermahnung ausgeprochen werden:
	\begin{itemize}
		\item
		Experimentiere nicht mit der Netzspannung, sondern nur mit
		Haushaltsbatterien!
		\item
		Schaue nie mit Lupen, Fernrohren oder \"{a}hnlichen optischen
		Ger\"{a}ten in die Sonne!
		\item
		Experimente, die du in der Schule siehst, k\"{o}nnen sich
		gef\"{a}hrlich auswirken, wenn sie zuhause wiederholt werden.
		\item
		Beim \"{O}ffnen von Elektroger\"{a}ten muss der Netzstecker gezogen sein!
		\item
		Sei vorsichtig mit Laserpointern, starken Federn, spitzen
		Gegenst\"{a}nden!
		\item
		Beim Experimentieren soll nicht gegessen oder getrunken werden.
	\end{itemize}

\end{itemize}

\begin{uea}
	Beurteilen Sie: Handelt es sich hierbei um Freihandexperimente?
	\begin{enumerate}[label=\alph*.]
			\item
		Das Modell eines Ottomotors oder einer Dampfmaschine
		wird demonstriert.
		\item
		Das Bild eines Regenbogens oder eines gl\"{u}henden Lavaflusses
		wird gezeigt.
		\item
		Die Sch\"{u}ler betrachten ein St\"{u}ck Kork unter dem Mikroskop.
		\item
		Ein Wagen bewegt sich gleichf\"{o}rmig auf der Luftkissenfahrbahn.
		\item
		Ein Sch\"{u}ler schleppt eine mit Steinen beladene Tasche in den
		4.\ Stock des Schulhauses.
		\item
		Ein Bolzen wird mit dem speziellen Apparat gesprengt. 
	\end{enumerate}
\end{uea}

\subsection{Logistik}

\begin{itemize}
	\item Paradiesische Zust\"{a}nde im Physiksaal ---  anders als in der Wirklichkeit.
	\item Aufbewahrung.
	\item Wesentliche Dinge vielleicht in einer Art Koffer (Mit 100 Teilen 1000 Experimente).
\end{itemize}

\subsection{Nachteile}

Es entstehen unter Umst\"{a}nden falsche Vorstellungen davon,
was Physik als Wissenschaft leistet
(und damit: \dots Physiker als Wissenschaftler leisten).

Spannungsfeld: Rechenphysik --- Frei gestaltete Physik.
