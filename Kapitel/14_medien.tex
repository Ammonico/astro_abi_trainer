\chapter{Medien im Unterricht}\label{Medien}

\section{Begriffsbildung}
Engerer (praktischer) Begriff (Fr\"{o}hlich, 1974): Medien sind
\begin{itemize}
\item
technische Unterrichtshilfen, die der Lehrer einsetzt oder
\item
Lernmittel f\"{u}r die Hand der Sch\"{u}ler.
\end{itemize}

Im weiteren Sinne k\"{o}nnte man als Medium all das bezeichnen, was der
Informations\"{u}bertragung und -bewahrung dient:
Die Luft als Tr\"{a}germedium
des Schalls und des Lichts, das gesprochene Wort \dots.

\subsection{Ziele beim Einsatz von Medien}
\begin{itemize}
\item
Informationsvermittlung und -bewahrung.
\item
Veranschaulichung: ,,Ein Bild sagt mehr als tausend Worte'',
Modellger\"{a}te, Modellversuche.
\item
Entlastung des Lehrers von Routine\-t\"{a}tigkeiten zugunsten einer
p\"{a}dagogischen T\"{a}tigkeit.
\item
Lebensn\"{a}he.
\item
Differenzierung im Klassenunterricht, Steuerung
im Gruppenunterricht.
\item
Vielf\"{a}ltige Variation des Unterrichtsgeschehens.
\end{itemize}

\subsection{\"{U}berblick \"{u}ber audiovisuelle (AV-) Medien}

Rechner in Gestalt von Festanlagen oder Notebooks/Laptops \dots in Verbindung mit Beamern/Lautsprechern
sind universell einsetzbare audiovisuelle Ger\"{a}te.
Die klassischen Ger\"{a}te werden teilweise immer weiter verdr\"{a}ngt:

\begin{itemize}
\item
Tafeln aller Art:
\begin{itemize}
\item
Wandtafel (Kreidetafel).
\item
Lehrtafeln (Landkarten, Tabellen (PSE, Isotopen\-karte))
\item
Tuch\-hafttafel (Filz, Flanell) mit Kletthaftpl\"{a}ttchen.
\item
Magnet\-haft\-ta\-fel (Magnet\-optik, Magnet\-haken)
\end{itemize}

\item
Projektoren aller Art:
\begin{itemize}
\item
Tageslichtprojektor (TLP oder OHP),
\item
Video-Anlage, Diaprojektor, Filmprojektor, Episkop sind angesichts der Pr\"{a}senz von Computertechnologie pratkisch verdr\"{a}ngt. \end{itemize}
\item
Audio-Ger\"{a}te aller Art:
\begin{itemize}
\item
CD-Player (Akustik, Schwebungen).
\item
Tonbandger\"{a}t, Kassettenrekorder, Plattenspieler, Radioger\"{a}t (Schulfunk) finden sich heute kaum noch in der Schulpraxis.
\end{itemize}
\end{itemize}


\subsection{Die Wandtafel}
\begin{itemize}
\item Mittelpunkt: Die Tafel ist auch heute noch das Medium, das im
Mittelpunkt des Unterrichts steht,
symbolisch entspricht dies der Anordnung der Tafel in
der Mitte der Stirnseite des Klassenzimmers.

\item Dynamik: Das Tafelbild wird w\"{a}hrend einer Stunde entwickelt
und spiegelt daher die Dynamik eines Unterrichts\-verlaufs wider.

\item Planung: Meist wird es mehr oder weniger genau --- eventuell
ma{\ss}st\"{a}blich --- geplant.
Es enth\"{a}lt dann in m\"{o}glichst pr\"{a}gnanten, klaren Formulierungen einen
auch f\"{u}r den Sch\"{u}ler-Hefteintrag geeigneten Lehrtext
und zugeh\"{o}rige Skizzen, Zeichnungen oder Diagramme.

\item Spontaneit\"{a}t: Es kann aber auch spontan entwickelt werden
(Nebenrechnungen, Skizzen, Kurzerkl\"{a}rungen, Messwerte,\dots).
\end{itemize}

\bip
Hinweise zur Arbeits\-technik  an der Wandtafel:
\begin{itemize}
\item
Die Tafel sollte mit einem sauberen Lappen gewischt und dann
trocken sein.
\item
Achte auf gute Sichtverh\"{a}ltnisse: Beleuchtung der Tafel
m\"{o}glichst von vorne oben, Unterbindung von Reflexionslicht.
\item
Achte bei der Planung auf Eigenschaften der Tafel:
Gr\"{o}{\ss}e der Teiltafeln, Klappm\"{o}glichkeiten, Skalierung der K\"{a}stchen,
Befestigungsm\"{o}glichkeiten (Haken, Stative, Magnetclips)
Korrektur nach Trockenwischen, Annahme von Farbkreide.
\item
Soll die Tafel (grunds\"{a}tzlich) bez\"{u}glich verschiedener
Funktionen unterteilt werden: Konzepttafel, Nebenrechnungen,
Hefteintrag,\dots

\item
Spreche nicht zur Tafel! Stehe seitlich zur Schreibhand!
Quietschende Kreide sollte durchgebrochen werden.
\item
Achte auf den Zeitbedarf!
Das Tempo beim Tafelschreiben ist von selbst eher sch\"{u}lergem\"{a}{\ss}.
\item
Zeichnungen:
\begin{itemize}
\item
Beim Zeichnen von Kurven ist die Richtung von oben nach
unten g\"{u}nstiger!
\item
Die Hand sollte ---  unverkrampft --- in etwa senkrecht
zur momentanen Strichrichtung ausgerichtet sein.
\item
Arbeite m\"{o}glichst mit Hilfspunkten und/oder Symmetrien.
\item
Dr\"{u}cke Zeichenwerkzeuge immer in der N\"{a}he des aktuellen
Zeichnens an!
\item
Bei perspektivischen Darstellungen ist \"{U}bung (evtl.\ vorher
auf dem Papier) von Vorteil.
Entwickle die Darstellung bez\"{u}glich der Richtung senkrecht zur
Tafel ,,in die Tafel hinein'' (wegen der Verdeckungsrelationen)!
\mip
\"{U}bung: Zeichne einen Hufeisenmagneten, eine Spule
mit Andeutung der Windungsrichtungen, einen Quader (W\"{u}rfel),
eine Kugel.
\end{itemize}
\item
Falsches (Beispiele f\"{u}r typische Fehler) sollte als solches
kenntlich sein.
\end{itemize}

Spezielle Hinweise f\"{u}r Mathematik und Physik:
\begin{itemize}
\item
Der Kreidezirkel sollte in der N\"{a}he der Saugn\"{a}pfe
festgehalten werden.
Der Kreismittelpunkt muss vor dem Kreiszeichnen markiert werden
(Es gibt keinen Einstichpunkt).
\item
Beim Zeichnen von geometrischen Figuren oder Graphen
(Messwerte, Funktionsgraphen) arbeite man mit Hilfspunkten oder
Symmetrien! Werden Punkte eines Graphen mit $+$ statt $\times$ markiert,
so sind sie nach der Fertigstellung des Graphen leichter
erkennbar.
\item
Bei einem drei-dimensionalen  Koordinatensystem sollte die
zur Tafelebene senkrechte Achse nach vorne gezeichnet sein.
Dann wird sie nicht von Objekten im Quadranten der anderen beiden
Achsen verdeckt.
\item
Verwendung von ,,Icons'':
Auge, Mikroskop, Motor,\dots

\item
Verwendung von Symbolen: Schaltbilder von el.\ Schaltungen,
Befestigungen.
\end{itemize}

\subsection{Der Tageslichtprojektor (TLP OHP)}

\pph{Optisches Prinzip}
Im Kopf des TLP befindet sich eine Sammellinse (hoher Qualit\"{a}t) und ein Spiegel.
\"{U}ber diese Anordnung wird der Gegenstand (die leuchtende Schreibfl\"{a}che) auf die Projektionswand abgebildet.
Im Prinzip ist das das gleiche wie bei der Abbildung mittels einer Sammellinse $(*)$ oben.
\mip
Unter der Schreibfl\"{a}che befindet sich eine leistungsstarke Lampe zur Durchleuchtung der Schreibfl\"{a}che.
Eine Fresnel-Sammellinse (geringe Qualit\"{a}t) sorgt daf\"{u}r, dass m\"{o}glichst viel Licht von der Lampe aus
durch die Schreibfl\"{a}che in Richtung des Abbildungskopfes gerichtet wird.

\pph{Einsatz als Schreibprojektor}
Es wird --- w\"{a}hrend des Unterrichts --- auf leere
transparente Einzelfolien oder Endlosfolien geschrieben.
\begin{itemize}
\item
G\"{u}nstig (als Rechtsh\"{a}nder) steht man links vom Ger\"{a}t, den
Blick in die Klasse gerichtet.
\item
Man richte anf\"{a}nglich die Projektion so ein, dass die ausgeleuchtete
Fl\"{a}che an der Wand oder auf dem Schirm vollst\"{a}ndig von den Sch\"{u}lern
eingesehen werden kann.
Die Oberkante der Leuchtfl\"{a}che sollte unterhalb der Schirm- oder
Deckenkante sein. Danach wird die Folie ausgerichtet.
Dann ist nur noch eine gelegentliche Kontrolle notwendig.
\item
Man achte auf die richtigen Beleuchtungsverh\"{a}ltnissse.
\item
Zeigende Erl\"{a}uterungen f\"{u}hrt man auf der Schreibfl\"{a}che, nicht auf
dem ausgeleuchteten Schirm durch. (Das ist gew\"{o}hnungsbed\"{u}rftig.)
\item
Bei l\"{a}ngerer Unterbrechung der Arbeit am TLP, insbesondere
beim Wechsel des Mediums, sollte man die Lampe ausschalten.
\item
Zum Schluss sollte der Klappspiegel zum Schutz vor (Kreide-)Staub geschlossen werden.
Da die Lampe im hei{\ss}en Zustand empfindlicher gegen\"{u}ber mechanischen Einfl\"{u}ssen ist, sollte
der TLP erst nach Abk\"{u}hlung der Lampe transportiert werden.
\end{itemize}

Die Funktion als Schreibprojektor kommt der der
Wandtafel fast gleich.
Vor- und Nachteile gegen\"{u}ber der Wandtafel sind \dots
\begin{enumerate}
\item Der Lehrer schaut zur Klasse.
\item Das Geschriebene ist konserviert und kann
ausgewertet, archiviert oder wiederverwendet (Wiederholung) werden.
\item Unendliche Kapazit\"{a}t
\item Fertige Transparente k\"{o}nnen eingebracht werden, z.B.\ als
,,Untergrund'':
Koordinatensystem, Leertabelle, L\"{u}ckentext.
\item
Folien k\"{o}nnen gedreht oder gewendet werden.
\item
Die Farbgebung ist einfacher.
Die Zuordnung hell/dunkel bzgl.\ des Sch\"{u}lerhefteintrags stimmt
(dunkel auf der Folie $\Rightarrow$ dunkel im Heft).
\end{enumerate}

\begin{enumerate}
\item
Die Lichtverh\"{a}ltnisse m\"{u}ssen g\"{u}nstig sein, es ist evtl.\ eine
Teilverdunkelung notwendig.
\item Der TLP ist als technisches Ger\"{a}t bedienungsunfreundlicher
(Einschaltknopf, Spiegel \"{o}ffnen, Fokussierung, Stromanschluss),
fehleranf\"{a}lliger
(Lampe) und teuer (OHP-Stifte, Folien).
\item
Das Tun des Lehrers ist nicht so gut erkennbar
(z.B.\ bei Konstruktionen).
Zirkel ist nicht verwendbar (Aber: Zirkel-GEO-Dreieck).

\item
Die Projektionsfl\"{a}che ist beschr\"{a}nkt (Problem bei Querformat).
\end{enumerate}

\pph{Einsatz als Transparentprojektor}
(Beliebt in der Wirtschaft, Management: ,,Pr\"{a}sentation''.

Es k\"{o}nnen fertige Transparente mit Texten, Skizzen, Tabellen oder
Diagrammen gezeigt werden.

\mip
Es ergeben sich viele M\"{o}glichkeiten, dynamische Abl\"{a}ufe (Bewegungen)
durchzuf\"{u}hren:
\begin{itemize}
\item Drehen oder Wenden,
\item Zerschneiden oder Zusammenst\"{u}ckeln von Folien,
\item \"{U}bereinanderlegen, Verschieben, Einklappen (Tesafilm),
Gegeneinanderdrehen (Druckknopf) von mehreren Folien.
\end{itemize}

\pph{Projektion von Experimenten}
\begin{itemize}
\item
Mit dem TLP steht grunds\"{a}tzlich eine intensive
Lichtquelle zur Verf\"{u}gung.
\item
Der TLP kann als Projektionslampe zum Schattenwerfen dienen:
Kreisbewegung wird als harmonische Schwingung gesehen.
Schatten einer Kerzenflamme.
\item
Anzeigeger\"{a}te mit transparenter Skala (Messinstrumente, Kompass)
\item
Licht-Schatten-Ph\"{a}nomene
\item
Mechanisch dynamische Vorg\"{a}nge: Sto{\ss}en, Ablenken,
Stahlnagel-Elektromagnet, Oersted-Versuch.
\item
Farbmischung (Subtraktiv bei farbigen Folien)
\end{itemize}


\bip\bip
\section{Schriftliche Medien --- f\"{u}r die Hand der Sch\"{u}ler}
\subsection{Das Schulbuch}

\subsection{Arbeitsbl\"{a}tter und -hefte}
\begin{itemize}
\item
Selbstgestaltet --- vorgefertigt.
\item
Inhalte k\"{o}nnen korrekt und auf den Punkt gebracht
dargestellt werden.
\item
Gestaltung individuell: Tabellen, Diagramme, Bilder, Skizzen.
\item
M\"{o}gliche Sch\"{u}leraktivit\"{a}ten (Begleitet auf der Folie):
\begin{itemize}
\item
Erg\"{a}nzung: L\"{u}ckentexte, Tabellen, Zeichnungen.
\item
Farbige Gestaltung: Texte markieren, Skizzen f\"{a}rben,
           Zeichnungs\-tei\-le kennzeichnen.
\item
Beschriften von Zeichnungen, Diagrammen.
\item
Grafische Gestaltung: Unterstreichen, Einrahmen, Schraffieren.
\end{itemize}

\item
\"{O}konomie: Zeitersparnis im Unterricht, in der Vorbereitung.
\item
Arbeitsblatt-Verwaltung durch den Lehrer:
Heute mit PC m\"{o}glich und vielf\"{a}ltig.

\item
Probleme:
\begin{itemize}
\item
,,Zettelwirtschaft'' bei Sch\"{u}lern (G\"{u}nstig: Nummerierung, Datumsangabe (Kieserblock).
\item
Eine beim Lehrer hervorgerufene ,,Stoffabarbeitungs-Philosophie''
korrespondiert mit einem Minimal-Aufwand an Vorbereitung.
Im Unterricht ruft dies einen Eindruck von Eint\"{o}nigkeit hervor.
(Es gibt Verlage, die genau diese Art von Lehrermentalit\"{a}t bedienen:
,,Da brauchen Sie nur noch in der Fr\"{u}h schnell kopieren''.)
\end{itemize}
\end{itemize}

\subsection{Das Sch\"{u}lerheft}

Praktische Hinweise f\"{u}r Zeichnungen:
\begin{itemize}
\item
Beim Zeichnen von Kurven/Graphen setze den Handballen in der
N\"{a}he auf und ziehe den Strich in Schreibrichtung!
Eventuell sollte vorher das Heft gedreht werden.

\item
Zeichnungen sollten nach M\"{o}glichkeit mit
Bleistift angefertigt werden,
da dann das Radieren m\"{o}glich ist.
(Auch wegen Verdeckungen muss manchmal radiert werden.)
\end{itemize}



\bip\bip
\section{Rechner\-einsatz --- speziell im Physikunterricht}

Es gilt zun\"{a}chst abzukl\"{a}ren, welche Funktionen einem Rechner im Unterrichtsgeschehen insgesamt zukommen k\"{o}nnen.

\subsection{Der Rechner als AV-Medium}

Pr\"{a}sentation von AV-Medien aller Art:
\begin{itemize}
\item Bilder, Filme
\item Unterst\"{u}tzend bei Pr\"{a}sentationen
\item Dynamische Ver\"{a}nderung
\end{itemize}


\subsection{Der Rechner als TOPIC}

Computer und Informationstechnologie pr\"{a}gen und ver\"{a}ndern die kulturelle,
wirtschaftliche, wissenschaftliche und technologische Lebenswelt
in gravierender Weise. Die Schule muss in ihrem Bildungs- und
Erziehungsauftrag dieser Entwicklung Rechnung tragen:
\mip
Computer sind Inhalte von Unterricht.
\begin{itemize}
\item Physikalische, mathematische, ,,informatische'' Grundlagen von Computertechnologien.
\item Grundlegende Hardware-Komponenten des Rechners und der wesentlichen Peripherieger\"{a}te.
\item Software:
\begin{itemize}
\item
Betriebssysteme, Oberfl\"{a}che.
\item
Grundlegender Aufbau und Bedienung von Programmen im allgemeinen:
Standards, Men\"{u}system, Dialogbetrieb, Fenstertechnik,
programm\"{u}bergreifende Bedeutung bestimmter Bedienungselemente
Tasten (Esc, Enter, Cursor, Tab, Funktionstasten), Mausfunktionen,
\item
Kennenlernen von Grundtypen handels\"{u}blicher Programme:
\begin{itemize}
\item Textverarbeitung,
\item Tabellenkalkulation (Excel,\dots)
\item Datenbanken, Expertensysteme,
\item Statistik,
\item Graphikprogramme, Gestaltung, Technisch Zeichnen, CAD,
\item Kommunikation (Internet, email)
\end{itemize}
\end{itemize}

\item Auswirkungen von Computertechnologien:
\begin{itemize}
\item Ver\"{a}nderungen im individuellen Leben: Medienerziehung, Computerspiele, Multimedia.
\item Ver\"{a}nderungen im Allgtagsleben: Bank, Versicherung, Reiseb\"{u}ro, Arztpraxis, Kommunikation.

\item Ver\"{a}nderungen im Arbeitsleben:
\begin{itemize}
\item
Rationalisierung: Steigerung der Produktivit\"{a}t, Entfall von
Arbeitspl\"{a}tzen, Problematik von Heimarbeitspl\"{a}tzen.
\item
Abh\"{a}ngigkeit von Technologien (vgl.\ Stromausfall)
\item
Befreiung von monotoner (geistiger) Arbeit.
\end{itemize}

\item Ver\"{a}nderungen in der Organisation einer Gesellschaft: Verwaltung, Datensicherheit, Datenschutz.
\item Weltweite Kommunikation: Internet, Email.
\item Information als Wirtschaftsgut: Inflation, Monopolisierung, Freier Zugang.
\end{itemize}
\end{itemize}

\subsection{Der Rechner als ,,Lernender'' (TUTEE)}

\begin{itemize}
\item
Hier tritt die Programmierm\"{o}glichkeit in den Vordergrund. Es l\"{a}sst
sich der Kern der Idee eines Algorithmus herausarbeiten.
(Organisatorisch wird dieser Aspekt im Fach Informatik oder in
entsprechenden Wahlkursen behandelt).
\item
Anspruchsvolle TUTOR-Programme beinhalten die Lernf\"{a}higkeit des
Tutors. So werden beispielsweise h\"{a}ufige Fehler registriert und in
speziellen Routinen behandelt.
\end{itemize}

\subsection{Der Rechner als Werkzeug/Hilfsmittel (TOOL)}

Computer bestechen vor allem durch die M\"{o}glichkeit, verschiedene
Funktionen zu kombinieren.
\mip
Der Begriff ,,Werkzeug'' relativiert den Stellenwert
des Computers: Erst in der Hand des anwendenden Menschen
entfaltet es seine Wirkung.

\begin{itemize}
\item
Erstellung von Produkten (Dokumente, Texte, Graphiken, Multi-Medien).
\item
Ausf\"{u}hrung algorithmischer Arbeiten: Numerisches Rechnen
(Taschenrechner, Einfachste PC-Programme), Schulwerkzeug zum
Rechnen oder Plotten  (Mathe\-Ass), Algebrasysteme (DERIVE,
Maple, Mathematica), Geometriesysteme (ZuL von Rene Grothmann/Eichst\"{a}tt, EUKLID, THALES, CA\-BRI-\-Geo\-metre)
\item
Auskunftssytem: W\"{o}rterbuch (fremdsprachlich, Rechtschreibung,
Grammatik, Fachlexika, Formelsammlungen.
\item
Ersatz motorischer T\"{a}tigkeiten (von Handarbeit): Konstruieren in
der Geometrie (Cabri Geometre, Thales, Euklid), Technisches
Zeichnen ($\to$ Architektur, In\-ge\-nieur-Berufe), CAD.
\item
Datenbank: Z.B.\ Abrufen und Ver\"{a}ndern von Arbeitstexten (alle
F\"{a}cher: D:Li\-te\-ra\-tur, R: Bibel, Ku: Gem\"{a}lde, Wi:
Statistiken,\dots)
Datenverwaltung: Z.B.\ Anlage eines Vokabelheftes. \\
Es k\"{o}nnen gro{\ss}e Datenmengen verarbeitet werden.
\item
Visualisierung: Geometrische Situationen (ebene oder r\"{a}umliche
Geometrie, Funktionsgraphen) k\"{o}nnen dargestellt, ver\"{a}ndert,
dynamisiert, animiert werden. (Pestalozzi: Ein Bild sagt mehr als 1000 Worte.)
\item
Vielseitiges Hilfsmittel f\"{u}r K\"{o}rperbehinderte.
\item
Vielseitiges (integriertes) AV-Medium (Pr\"{a}sentationen),
MULTIMEDIA, LiveCam, Scanner.
\item
Simulation dynamischer Abl\"{a}ufe (Ph: Schwingungen, Radioaktiver Zerfall,
Sk: Alterspyramide, B: Populationen, C: Reaktionen, M: Fraktale).
\item
In Kombination mit Peripherieger\"{a}ten aller Art: Erfassung von
Messwerten oder Musik-Kompositionen. Steuerung von Modellen,
Robotern, Musikinstrumenten.

\item
Messwert-Erfassung bei naturwissenschaftlichen Experimenten: Vergleiche weiter unten.
\end{itemize}

\subsection{Der Rechner als Lehrender (TUTOR)}
Der Computer als automatisierter \"{U}bungspartner. Dabei k\"{o}nnen als Sozialformen
Einzel- oder Kleingruppen-Unterricht mit oder ohne
Wettbewerbssituation realisiert werden. Vorteil:
Schwierigkeitsgrad, Intensit\"{a}t der F\"{u}hrung individuell
einstellbar, variierbar (Zufallsgenerator).

\begin{itemize}
\item
Abfragen: Vokabeln, Rechtschreibung, Grammatik,
Wortschatz (Thesaurus), Formeln.
\item
Ein\"{u}bung von Algorithmen: Schriftliche Rechenverfahren, einfache
bis komplexe sbungsaufgaben.
\item
Extremform: Programmierter Unterricht.
\end{itemize}


\subsection{Der Rechner als Hilfsmittel im Lehrer-Beruf}

\begin{itemize}
\item
Erstellung von Arbeitsbl\"{a}ttern, Erstellen oder Austesten von
Pr\"{u}fungsaufgaben,
\item
Verwaltungsfunktionen (Notenverwaltung),
\item
Auseinandersetzung mit dem Computer im Unterricht.
\end{itemize}

\subsection{Einsatz in der Schulverwaltung}

\begin{itemize}
\item
B\"{u}roarbeiten (Texte, Tabellen,\dots)
\item
Verwaltung von Sch\"{u}lerdaten (Noten, Zeugnisse, Adressen,
Schulkarriere), statistische Erhebung und Auswertung.
\item
Stundenplan, Vertretungsplan.
\item
Mittelverwaltung.
\item
Kommunikation.
\end{itemize}

Wichtig dabei ist ein durchdachtes Konzept zur Trennung von Rechnern f\"{u}r Schulverwaltung und
Unterricht.

\subsection{Messwert-Erfassung bei physikalischen Experimenten}

Die USB-Technologie erm\"{o}glicht eine sehr flexible und einfach zu bedienende Kombination von
\begin{itemize}
\item Sensoren f\"{u}r vielf\"{a}ltige physikalische Gr\"{o}{\ss}en \q und
\item Verarbeitung, Speicherung, Darstellung von Messergebnissen.
\end{itemize}

Es gibt Sensoren f\"{u}r:
\begin{itemize}
\item Bewegung (Schall- oder IR-Reflexion),
\item Lichtschranken, auch in Verbindung mit Bewegungs-Messwandler \\
(BMW: Die Bewegung eines Speichenrades wird durch die Lichtschranke aufgenommen),
\item Rotationsbewegung,
\item Kraft,
\item Druck,
\item Schall (Lautst\"{a}rke, Frequenz: Fourier-Analysator),
\item Stromst\"{a}rke, Spannung,\dots
\item el.\ Leitf\"{a}higkeit,
\item Ladung,
\item Magnetfeld,
\item Temperatur,
\item Lichtintensit\"{a}t,
\item pH-Wert.
\end{itemize}

Vgl.\ zum Beispiel das System der amerikanischen Firma PASCO.
\mip
Verarbeitung von Messergebnissen:
\begin{itemize}
\item Algebraische Kombinationen, Ableitung oder Integral.
\item Vergleich mit der Literatur.
\end{itemize}

\mip
Speicherung von Messergebnissen:
\begin{itemize}
\item Fixierung, Speicherung, Verarbeitung, Sicherung riesiger Datenmengen.
\item Leichte Portabilit\"{a}t.
\item Vorhandene Daten k\"{o}nnen zur wiederholenden Simulation genutzt werden.
\end{itemize}

\mip
Darstellung oder Pr\"{a}sentation von Messergebnissen:
\begin{quote}
Analog --- digital --- tabellarisch --- graphisch.
\end{quote}
Die analoge Darstellung der digitalen Daten wird dabei simuliert.

\mip
Mit Beamer oder White-Board k\"{o}nnen Messergebnisse simultan von allen Sch\"{u}er(inn)en beobachtet werden.

\mip
Paralleldynamik: Das Experiment wird parallel sowohl real als auch modellhaft-virtuell im Rechner durchgef\"{u}hrt.
So k\"{o}nnen vorteilhafte Aspekte des realen Experiments (Unmittelbarkeit) und der Modellierung im Rechner (Herausarbeitung
wesentlicher Aspekte, Idealisierung) kombiniert werden.

\pph{Vorteile des Einsatzes von computerbasierter Messwerterfassung}
\begin{itemize}
\item Schnelle, bequeme Bedienbarkeit $\to$ Zeitersparnis.
\item Vielseitigkeit: Vergleiche oben.
\item \"{U}bersichtliche Aufbewahrung $\to$ Platzersparnis.
\item Standardisierung,
\item Lernziel: Einblick in anders geartete heutige M\"{o}glichkeiten f\"{u}r den Computereinsatz.
\item Lernziel: Einblick in heutige Elektronik-Technologien zur Erfassung physikalischer Gr\"{o}{\ss}en.
\item Blackbox-Prinzip: Der wesentliche Gehalt eines Experiments kann besser herausgearbeitet werden.
\end{itemize}

\pph{Nachteile des Einsatzes von computerbasierter Messwerterfassung}

\begin{itemize}
\item Die Unmittelbarkeit der messenden Erfassung von physikalischen Ph\"{a}nomenen geht verloren.
\item Handwerkliche Fertigkeiten werden in den Hintergrund gedr\"{a}ngt.
\item Zunehmende Abh\"{a}ngigkeit von ausgefeilten Technologien.
\item Kosten f\"{u}r Neuanschaffungen. Eigentlich sind Messwerterfassungssysteme vergleichsweise billig.
\end{itemize}


\bip\bip
\section{K\"{u}nstliche Intelligenz \`{a} la ChatGPT --- speziell im Physikunterricht}

ChatGPT und verwandte Angebote zur Nutzung k\"{u}nstlicher Intelligenz sind erst k\"{u}rzlich allgemein zug\"{a}glich geworden, erfreuen sich aber bereits allergr\"{o}{\ss}ter Beliebtheit insbesondere bei Sch\"{u}lerinnen und Sch\"{u}lern, sowie Studenten. Die Entwicklung der k\"{u}nstlichen Intelligenz steht noch am Anfang, sodass damit zu rechnen ist, dass sich die zur Verf\"{u}gung stehenden M\"{o}glichkeiten und die Qualit\"{a} der Ergebnisse stetig weiterentwickeln werden. Auf eine Darstellung, was ChatGPT ist und wie man es benutzt, wird hier verzichtet. Vielmehr sollen hier Ideen skizziert werden, auf welche Weise ChatGPT gewinnbringend im Physikunterricht genutzt werden kann.

\begin{enumerate}
\item \textbf{Generieren von Inhalten:} ChatGPT kann mit gro{\ss}er Leichtigkeit Texte und eine Vielzahl von Grafiken effizient erzeugen. 
	\begin{itemize} 
	\item Beispiel: \say{\emph{Erstelle mir eine kurze Zusammenfassung zum Film 'The Fast and the Furious', die ich in einer dreimin\"{u}tigen Pr\"{a}sentation in der Schule verwenden kann, im Sprachstil von Letty Ortiz!}}
	\end{itemize}

\item \textbf{Study-buddy:} Sch\"{u}ler k\"{o}nnen  mit ChatGPT interaktiv und im eigenen Tempo \"{U}bungsfragen durchgehen und Rechenaufgaben l\"{o}sen. Physikalische Konzepte k\"{o}nnen durch ChatGPT verst\"{a}ndlich erkl\"{a}rt werden und im Dialog mit ChatGPT erlernt werden. Im Gegensatz zu Google ist KI in der Lage, einen Dialog mit dem Bedienenden zu f\"{u}hren, auf Fragen zu antworten, auf Nachfrage zu pr\"{a}zisieren, Beispiele zu liefern, Gegenbeispiele zu liefern usw.. ChatGPT kann somit als Gespr\"{a}chspartner in simulierten Diskussionen \"{u}ber physikalische Themen dienen. 
	\begin{itemize}
	\item Beispiel: \glqq\emph{Nenne mir ein Experiment, mit dem nachgewiesen wird, dass die Ladung des Elektrons wirklich negativ ist!}\grqq,  analysiere die generierte Antwort kritisch, und stelle Folgefragen, beispielsweise \glqq\emph{Wird durch dieses Experiment wirklich das Vorzeichen der Ladung ermittelt?}\grqq.  
	\end{itemize}

\item \textbf{Ideengeber bei Hausaufgaben und Projekten:} Sch\"{u}ler k\"{o}nnen ChatGPT nutzen, um Unterst\"{u}tzung bei Hausaufgaben, Projektideen oder der Recherche nach Informationen zu physikalischen Themen zu erhalten. 
	\begin{itemize} 
	\item Beispiel: \glqq\emph{Nenne mir drei Experimente zum Nachweis des Strahlencharakters von Licht f\"{u} den Physikunterricht der 8. Klasse!}\grqq
	\end{itemize}

\item \textbf{Begutachtung von schriftlichen Arbeiten und Vorbereitung auf Pr\"{u}fungen:} Sch"{u}ler k\"{o}nnen ChatGPT nutzen, um sich auf Pr\"{u}fungen vorzubereiten, indem sie Fragen stellen, Erkl\"{a}rungen wiederholen und gezielte R\"{u}ckmeldungen zu ihren Antworten erhalten. ChatGPT kann als virtueller Gutachter verwendet werden, um eigene Textentw\"{u}rfe zu begutachten. 

	\begin{itemize} 
	\item Beispiel: Ziehen Sie eine pdf-Datei mit einer selbsterstellten Ausarbeitung oder auch eine Grafik in den Eingabeprompt und bitten Sie um eine Zusammenfassung oder eine Analyse bzw. Bewertung!
	\end{itemize}
	
	
\item \textbf{Unterst\"{u}tzung f\"{u}r Lehrkr\"{a}fte:} Lehrer k\"{o}nnen ChatGPT als Hilfsmittel bei der Unterrichtsvorbereitung einsetzen, z.B. zum Erstellen von \"{U}bungsaufgaben, zur schnellen Recherche oder als Inspirationsquelle f\"{o}r neue Unterrichtsideen. 
	\begin{itemize} 
	\item Beispiel 1: {\glqq}\emph{Gib die erforderlichen Lernvoraussetzungen, Lernziele und eingesetzte Medien in einer Unterrichtsstunde zum Fadenstrahlrohr an.}{\grqq} 
	\item Beispiel 2: {\glqq}\emph{Gestalte eine sch\"{u}lerzentrierte Unterrichtseinheit zum Thema elektrische Influenz in der Mittelstufe. W\"{a}hle ein Artikulationsschema des entdeckenden Unterrichts. Gehe explizit auf Vorwissen und Lernziele ein und verdeutliche, wie durch die Ma{\ss}nahmen dieser Unterrichtseinheit die gesetzten Lernziele erreicht werden k\"{o}nnen.}{\grqq} 
	\end{itemize}
\end{enumerate}
\bip

ChatGPT erm\"{o}glicht  nicht nur effizientes und kreatives Arbeiten, ebenso kann ChatGPT  den Physikunterricht bereichern, indem es personalisierte Lernm\"{o}glichkeiten bietet  und den Zugang zu komplexen Inhalten vereinfacht. Durch die Verwendung von ChatGPT wird das selbstgesteuerte Lernen stark gef\"{o}rdert und eine Bef\"{a}higung zum zum kritischen und ausgiebigen Dialog angestrebt.
\mip
Es ist jedoch darauf zu achten, dass der Einsatz von ChatGPT bewusst gesteuert wird, um sicherzustellen, dass es als Erg\"{a}nzung und nicht als Ersatz f\"{u}r den direkten Unterricht und die zwischenmenschliche Interaktion dient. Beim Generieren von Inhalten wird die Eigenleistung der Lernenden im kreativen Recherche- und Schreibprozess umgangen. Lehrkr\"{a}fte m\"{u}ssen sich mit den M\"{o}glichkeiten durch ChatGPT vertraut machen und klare Regeln zu dessen Anwendung aufstellen.
\mip
Letztlich muss erw\"{a}hnt werden, dass ChatGPT gelegentlich zum {\glqq}Halluzinieren{\grqq} neigt, d.h. dass es frei erfundene Antworten erzeugt. Output von ChatGPT sollte daher niemals  ungepr\"{u}ft \"{u}bernommen werden. 

